\chapter{How to Use}
\emph{''Simple is perfect!''}

And it is modern.
\label{cha:commonusage}

\piccbc \newpage

\section{Framework}

Framework provides basic funcions that all pluses shared.

\begin{description}

\item[To Change Root Menu Position] Choose where LeXtudio menu should be.

Select ''Options $|$ Preferences'' menu item. In the dialog, click on
''Framework'' node on the left. In the right panel check the position you
prefer. Click OK to exit.

\pic{framework}{Preferences Form}{width=0.5\textwidth}

Restart the IDE, then the changes take effect.

\item[To Change Shortcuts] Choose another shortcut for a function.

Select ''Options $|$ Shortcuts'' menu item.

\pic{shortcuts}{Shortcuts Dialog}{width=0.5\textwidth}

In the dialog, choose a menu name and press the key combination you like. Click
OK and the changes take effect at once.

\boxing{You must notice that not all menu items are listed there. Only authors
can define which item has a shortcut, not users. Two many shortcuts may conflict
with each other.}
\end{description}


\section{CodeBeautifiers Plus}

This plus helps you beautify source files. It is also what CBC is named after.
It is inspired by the Code Beautifier feature in Sharp Builder Tools.

\begin{description}

\item[To Format A File] Format an open file.

\pic{beautify}{Beautify Menu}{width=0.25\textwidth}

Open a source file in IDE.

Click ''Beautify $|$ Document'' (or use its shortcut, Ctrl + W by default) to
beautify/format it. A message of ''File type is not supported'' is displayed in the
Message Panel if you try to format an unsupported file.

\item[Undo Formatting A File] Undo formatting.

The standard undo command, Ctrl + Z, is available, which brings back the
original file content.

\item[To Change Formatting Styles] Change the coding style of formatting.

\begin{description}

\item [Basic] Choose among the Built-in styles.

Select ''Options $|$ Preferences'' menu item. In the dialog, click on ''Object
Pascal'' or ''C/C++/C\#'' node on the left. In the right
 panel, check the style you prefer. Click OK to exit.

\pic{astylestyles}{AStyle Styles}{width=0.5\textwidth}



\pic{jcfstyles}{JCF Styles}{width=0.5\textwidth}

\item[Advanced(Experimental)] Command-line configuration or custom settings files.

You must feel sad that only a few built-in style
are available. Generally
speaking, it is very hard to fully configure the formatter, so built-in styles
are suggested. Changing the command line settings is not an easy task but they
provide you much more power. You should read the provided help files in the
''Help'' folder to learn about AStyle and JCF command line carefully.

Follow the Basic tip and choose command line style. An edit box is showed for
you to type in command line parameters. AStyle can be configured very well in
this way. However, for JCF I suggust you use its GUI version to modify the
settings file. If you have such a JCF settings file already, you can choose
''Custom File'' option in JCF tab page and input the file's path.

Since 6.0 a JCF Style Editor is added. You can now switch to Custom Style and
create your own Object Pascal coding style by clicking Configure button.
\end{description}

This setting takes effect immediately.

\end{description}

\section{Utilities Plus}

This plus contains basic utilities to configure CBC.

\begin{description}
\item[Tip of the Day] Display tips.

\pic{tipoftheday}{Tip of the Day}{width=0.5\textwidth}

\item[Expert Manager] Manage IDE experts, both Win32 and .NET.

\pic{expertmanager}{Expert Manager}{width=0.5\textwidth}

\item[Plus Manager] Manage CBC pluses and features.

\pic{plusmanager}{Plus Manager}{width=0.5\textwidth}

Checked features are enabled when IDE loads.

If a plus is checked, all its features are enabled. If a plus has all its
features enabled, it is checked, too.

% Partially enabled plus is displayed in yellow.

\boxing{Some experimental features are disabled when a stable version is
released. You may turn them on to have a preview.}

% \item[HelpWizards] Add some useful links under IDE Help menu.
%
% \pic{help}{Help Menu Items}{width=0.25\textwidth}
%
% LeXDK Developer's Guide and the help of CBC are links to PDF files. You must
% have a viewer installed before viewing them.
%
% Others are links to websites. There is a link to GForge homepage of CBC, and
% another to my weblog.

\item[InDate] Keep your CBC version up to date.

If a new version is available, it will prompt you to download and install it.
Otherwise, the dialog closes automatically.

\pic{indate}{InDate Dialog}{width=0.5\textwidth}


\end{description}

\section{AddMany Plus}

This plus enables you to add a lot of files to your project all at a time. This
plus is a port of AddMany 5.2.

\begin{description}
\item[To Add Many Files to Current Project] Add many files in different folders
to current project.

Click ''Add Many to Project\dots'' (or use its shortcut,
Ctrl + Alt + A by default\footnote{This conflicts GExperts, and users can
override it.}). A dialog pops for you to choose a folder.

\pic{browsefolder}{Folder Selection}{width=0.5\textwidth}

After selecting and click OK, all source files in this folder and sub-folders
are listed in a dialog for you to select.

Multi-selection in this dialog is similar to the same operation in Windows by
pressing Ctrl key while selecting or Shift key used to select a section of
files.

\pic{addmanydialog}{AddMany Dialog}{width=0.5\textwidth}

Click OK after selection, and those files are added to current project. Hints
are displayed in the Message Panel.

\item [To Add Another Project's Files] Add all source files of another project
to current project.

Choose a valid IDE project and click OK, the source files of the selected
project will be added to current project.

\boxing{As a limitation of AddMany 5.2, the related references are not added
automatically. You have to manually add them to your project.}

\end{description}

\section{ArtCSB Plus}

This plus adds a Designer Navigator and an Auto-Save utility. It is a port of
ArtCSB 2.5.

\begin{description}
  \item[Designer Navigator] It provides a structure view for WinForms and
  WebForms Form Designer like Visual Studio's Document Outline.

  \item[Auto-Save Utility] It auto-saves files in the editor according to your
  settings. In this way you can prevent data loss when IDE crashes. Because
  Delphi 2007 Release 2 already includes a similar feature, this utility is
  obsolete.
\end{description}

\section{BagPlug Plus}
This plus contains a few GPL covered features \lextm\ created.

\begin{description}

  \item[Readme Notifier] if your project contains a ''readme'' file, it will be
  opened automatically when the project is opened inside IDE.
  \item[Source Navigator] help to navigate along a long file. (Object Pascal
  only)
  \boxing{because Delphi CodeDOM has a different implementation, the
  first line after ''begin'' is highlighted. Also, global functions are not navigatable.}
  \item[Typing Speeder] Some templates are triggered by Space
key to speed up your typing.

in C\# file, type if and press Space, the rest of a code template is added
automatically for you. There are only a few templates now and cannot be
modified with a dialog.

\boxing{You may notice that BDS 2006+ has a feature named Live Code Templates.
In that case, Typing Speeder will add no more value.}

% An important function of Speeder is that when you type ''///'' and press Space
% before a declarartion (class, method, and so on), for example,
%
% \code{///(Press Space key here)\\
% public void GetBack() \{...\} }
%
% basic XML comments will be generated for you. At this moment, this function supports
% only C\# files. However, because of a .NET bug, it does not work for constructors.

\boxing{Sometimes Speeder fails because OTA has a bug.}

\item[Line Counter] provides simple statistics of the project group (solution)
based on line counting.
  \item[Icon Browser] It is a utility created by Kenny Kerr. You can browse all
  icons inside a file (exe, dll, bpl, and so on) and save them into .ico files.
\end{description}

\section{CSBuilderGoodies Plus}

This plus helps you preview C\# XML comments in a browser. It is a port of
C\#Builder Goodies 1.1.

\begin{description}
\item[To preview an XML Comment] View a section of XML comments in a browser dialog.

Open a C\# file and place the cursor inside an XML comment section. Select
''Doc Preview'' menu item (or use its shortcut, Alt + Q by default) to activate
the previewer.

\pic{docview}{Doc Previewer}{width=0.5\textwidth}

Click other forms, and this previewer will hide itself automatically.

\end{description}

\section{NFamily Plus}

This plus is experimental, so not documented here. It adds a few menu items that
help you call Expresso, FxCop, NUnit, and other .NET tools from with the IDE.

\section{WiseEditor Plus}

This plus adds a few interesting enhancement to the IDE. Since this plus is
under construction yet, many new features are not documented at this moment.

\begin{description}

\item[Editor/Project Aid Menu Items] Useful items.

\pic{edit}{Editor Aid Menu Items}{width=0.25\textwidth}

% \pic{filewizards}{File Wizards}{width=0.5\textwidth}

\item[File Aid] provide complex file searching functions.
\item[Clipboard Watcher] It tries to manage the clipboard for you.
\item[Color Pallette Helper] sometimes it is hard to remember a pre-define
System.Color name or ARGB value. This helper provides a Color Pallette to
choose the color visually and generate the name or the ARGB value for you.

Stop when you need to add a phrase such as Color.* or Color.FromArgb(****) and
click ''Color Pallette Helper'' menu item, a Color Dialog will show. Choose the
exact color you like and exit the dialog, the phrase will be generated for you.

\item[Shell Launchers] they help you to open a shell(cmd.exe) from common
places.

\item[Spell Checker] make a selection of words, and check their spellings.

Do a selection in the source editor and select ''Spell Check\dots'' menu item.
The dialog popped is similar to Microsoft Word's Spell Check dialog. You can
then decide whether to correct a spelling error or ignore it.

\pic{spell}{Spell Checker}{width=0.5\textwidth}


\item[WinMerge]WinMerge is a open source tool that can compare two files.

\item[Favorites]It provides a dialog to manage favorite files, projects, and references.

   \item[Extra] some tiny helpers.
\end{description}

