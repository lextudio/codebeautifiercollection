\chapter{What is Code Beautifier Collection 6}
\emph{''WHAT are the benefits that you want from a wizard?''}

This wizard has been evolved to make it one of the BEST since it was
born. \piccbc
\newpage

\begin{itemize}
\item Code Beautifier Collection 6\index{CBC} is a free and open-source expert
for CodeGear RAD Studio versions. Users needn't pay for this
useful tool.
	\begin{description}
% 	\item[Borland Developer Studio\index{IDE}] including C\#Builder (1.0), Delphi
% 8 for .NET (2.0), Delphi 2005 (3.0), BDS 2006 (4.0) and above.
	\item[CodeGear RAD Studio\index{RAD Studio}] including Delphi 2007, C++Builder
	2007, and RAD Studio 2007.
	\end{description}
\item It can do code beautifying/formatting automatically for source files in
all supported languages (currently five are available, Delphi/\-C/\-C++\-/C\#/\-XML). For example, a project group of both Delphi and C\# files can be formatted quickly without any tough setting ahead.

\item CBC is designed to be as simple as possible, as powerful as possible, and as convenient as possible. So both beginners and experts will find it usable.

\item Since version 5 many other useful features such as Typing Speeder,
AddMany are added. Details are listed in chapter \ref{cha:commonusage} on page \pageref{cha:commonusage}.
\end{itemize}


\pic{frontpage}{BDS Splash Form}{width=0.5\textwidth}
% \pic{menus}{CBC Menus}{width=0.25\textwidth}

%\begin{flushleft}
%\textbf{TIPS} you should use CVS tools if you can in order to get the newest version of these two.
%\end{flushleft}