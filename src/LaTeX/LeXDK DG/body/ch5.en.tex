
\chapter{Procedures}
\newpage

The minuses and pluses I wrote are used as the examples, such as the C\#Builder 
Goodies Plus (contained in Lextm.CSBuilderGoodies.Plus.dll), because they are 
the best demo I can give you. If you meet a special problem, you can contact me 
by sending a mail to \underline{lextudio@gmail.com}.
 
\section{How to Start a Plus}

Every plus project should contain at least one feature. In my plan, C\#Builder 
Goodies Plus only contains a QuickDoc Viewer feature (taken from C\#Builder 
Goodies 1.1).

\subsection{New Plus Project}
\index{Plus creation}
Make an empty .NET 1.1 library project in SharpDevelop (or any other C\# IDE).
Name it as Lextm.CS\-Builder\-Goodies.Plus.

\boxing{Plus is used as the last part so as to indicate it is a Plus project.
Lextm is used because \lextm\ is the author.}

\subsection{New Feature Class}
Add a public class named CSBuilderGoodiesFeature. Its base class should be 
CustomFeature.\index{feature creation}

\code{ public class CSBuilderGoodiesFeature : CustomFeature \{//...\} }

CustomFeature defined in LeXDK is the base class of every feature. It 
implements ILoadableFeature interface.

The class you created in last step is empty.

Since version 5.3.1.1124 of LeXDK, it is no longer needed to be made the class 
Singleton, but you have to make sure that it has a public constructor without 
arguments (this constructor will be later called by the Framework) :

\code{ public CSBuilderGoodiesFeature( ) \{//...\}}

Before going forward, several design decisions had better be made here:

\begin{description}
  \item[Menu Items] Menu items build up a bridge that connects a feature to 
  IDE. You may define necessary menu items in your feature and let LeXDK create 
  them for you.
  
  \item[Preferences] Usually a feature needs preferences management. You can 
  use the build-in service (defined in PropertyRegistry unit) or do it yourself.
  
  \item[Plus2 Registration File] It is required by LeXDK to register a feature
  and its plus. Since this file is read by the Framework, it is necessary.
  Pluses without registration files are considered invalid and not loadable.
  
  \boxing{Since a .plus2 file is a separate topic, details on it are placed in 
  appendix.}
\end{description}

These design topics are described in follwing sections.

\section{Define Menu Items}
\index{Menu item creation}

\code{ //Taken from CSBuilderGoodies Plus.\\
private const int DefaultViewDocShortcut = \\
\tab SharedNames.ALT + (int)Keys.Q;\\
private const string MenuViewDoc = "CBCExpertViewDocMenu";\\
private const string MenuLeXtudio = "CBCLeXtudioMenu";\\

private const string MenuTextViewDoc = "View Doc";\\

protected override void IDERegisterMenus()\\
\{\\

\tab base.IDERegisterMenus();\\

\tab RegisterMenu(\\
\tab \tab CreateActionMenu(\\
\tab \tab \tab MenuItemLocation.Child,\\
\tab \tab \tab MenuLeXtudio,\\
\tab \tab \tab MenuViewDoc,\\
\tab \tab \tab DefaultViewDocShortcut,\\
\tab \tab \tab MenuTextViewDoc,\\
\tab \tab \tab new EventHandler(DoViewDoc)\\
\tab \tab )\\
\tab );\\
\}\\

private static void DoViewDoc(object sender, EventArgs e) \{\\
\tab CsbGoodies.QuickdocViewer.Activate();\\
\}\\
}

There are four menu types you can define at this moment. They are ActionMenu, 
EmytyMenu, SeparatorMenu, and PlaceHolderMenu.
\begin{description}
  \item [Action Menu] A menu item that can do something (this item must have an 
  event handler).
  
  \item [Empty Menu] An example is the LeXtudio menu item. Click it, and no 
  action executes and a submenu is shown.
  
  \item [Separator Menu] It will be translated to a separator in the menu.
  
  \item [PlaceHolder Menu] It will not be translated to a real menu item. They 
  are place holder when you configure the menu tree storing the menu items. The 
  ROOT of this tree is a good example.
  
\end{description}

In above sample, C\#Builder Goodies Plus defines a ''View Doc'' action menu 
item. Let us see the steps.

At first, override a function named IDERegisterMenus, and new menu items are 
defined here (and created internally by LeXDK).

\boxing{Remember to call base function in the front (although the base function 
is currently empty) in case that I add some lines in future versions of LeXDK.}

RegisterMenu function call makes the menu item you define registered. Send a 
menu item definition as its parameter. Such a definition (of a specific menu 
type) can be created by calling a corresponding function, such as 
CreateActionMenu.

Here, a Doc Preview action menu item is defined by calling CreateActionMenu. It 
is a child of MenuLeXtudio, the first menu item of CBC. Its name is 
MenuViewDoc. Its shortcut is DefaultViewDocShortcut, and the text is 
MenuTextViewDoc. When you click the menu item finally in IDE, the handler 
DoViewDoc is called.

\boxing{There is no existing menu name list in this guide. However, you can 
read CBC source to find such items. Or you may use ''Mix for OTA'', a shortcut 
the CBC installer adds in your Start Menu.\\SharedNames class contains the 
most-frequently-used items such as Menu\-LeXtudio.\\According to Galileo OTA 
constraints, no menu name can contain dots. Otherwise, a run time error occurs.}

\boxing{Also you can see it is not hard to choose a shortcut value.\\SharedNames.ALT + (int)Keys.Q will give you the BDS TShortcut value for Alt + Q combination.}

Notice that if you don't give a shortcut to an item at design time, the users 
cannot add a shortcut to it later at run time. It is my intention to design 
LeXDK this way. Also, I suggest you add shortcuts to a very limited range of 
items. Too many shortcuts may collide.

\section{How to Manage Preferences for a Feature}
Now basic preferences management service is provided by LeXDK. Meanwhile, if 
you want advanced service, you can define and implement it, too.

\boxing{What is preferences? It is user options that need to be loaded when IDE 
starts and saved whenever necessary.}

Up to now, only simple type options can be managed by Property Registry. They 
are bool, int, double, string, enum, and so on, all of which can be 
de/serialized. If you look into the code, you will see most build-in features 
manage their options this way.

But if you have to use complex type options, such as list, array, table, and so 
on, you'd better manage them yourself at this moment (they cannot be 
de/serialized properly). FavoritesFeature and ComponentNamerFeature give good 
examples.

\subsection{Using LeXDK Property Registry}

Wherever you need to load a property, such as AStyle executable, you can call 
Property Registry. PropertyRegistry.Get(''AStylePath'') will give you the 
correct path if the value stored somewhere (remember to cast whenever 
necessary), and null is returned if there is no match. It is usually safer to 
call with a default value, such as PropertyRegistry.Get(''AStylePath'', 
''C:$\backslash \backslash$''), as the default value will be returned if there 
is no match is the registry. To save a property, simply use 
PropertyRegistry.Set(''AStylePath'', ''C:$\backslash \backslash$'').

\subsection{Advanced Management}
If you want to manage preferences yourself, you must override LoadPreferences, 
SetDefaultPreferences, and SavePreferences functions (defined in CustomFeature).

Generally speaking, you can do whatever sensible to manage the options, but for 
me it is easy to follow the method David used in Sharp Builder Tools.

Define a Preferences class (you can choose any name you like) and mark it as 
[Serializable] at first. The class must have a public constructor (.NET 
requirement). The following sample shows how ComponentNamerFeature class uses 
this service.

\code{ [System.Serializable]\\ public sealed class Preferences \{\\

\tab private ControlPrefix[] controlPrefixes;\\

\tab public ontrolPrefix[] ControlPrefixes \{\\
\tab \tab get	\{\\
\tab \tab \tab return controlPrefixes;\\
\tab \tab \}\\

\tab \tab set \{\\
\tab \tab \tab controlPrefixes = value;\\
\tab \tab \}\\
\tab \}\\

\tab public Preferences( ) \{	\} \\

\}\\

public static Preferences Options;

public override void LoadPreferences( ) \{\\
\tab base.LoadPreferences();\\
\tab Options = SerializationService.Load(this.GetType(),\\
\tab \tab typeof(Preferences));\\
\}\\

public override void SavePreferences( ) \{\\
\tab base.SavePreferences(); \tab SerializationService.Save(this.GetType(), \\
\tab \tab Options));\\
\}\\

public override void SetDefaultPreferences( ) \{\\
\tab base.SetDefaultPreferences();\\
\tab	if (options == null)\\
\tab	\{\\
\tab \tab		options = new Preferences();\\
\tab \tab		options.ControlPrefixes = new ControlPrefix[0];\\
\tab	\}\\

\tab	if (options.ControlPrefixes == null\\
\tab \tab			$||$ options.ControlPrefixes.Length == 0) \\
\tab    \{\\
\tab \tab       ArrayList list = new ArrayList();\\

\tab \tab        list.Add(new ControlPrefix("bt", "Button"));\\
\tab \tab        list.Add(new ControlPrefix("chk", "CheckBox"));\\
\tab \tab        list.Add(new ControlPrefix("cb", "ComboBox"));\\
\tab \tab        list.Add(new ControlPrefix("lv", "ListView"));\\
\tab \tab        list.Add(new ControlPrefix("txt", "TextBox"));\\

\tab \tab		options.ControlPrefixes =\\
\tab \tab \tab (ControlPrefix[]) list.ToArray(typeof(ControlPrefix));\\
\tab     \}\\
\}\\
}


Whenever you need to use the preferences object in your code, please refer to 
Options.

Sometimes you need to use Options somewhere outside the feature class, you want 
to find the feature object at first. It is easy to do this with the help of 
Feature Registry (defined in FeatureRegistry unit). Just call 
FeatureRegistry.GetFeature with the full qualified name of the feature class as 
the parameter. The object will be returned to you as ILoadableFeature. You can 
cast it to the type you need.

\subsection{Register a Tab Page to Global Preferences Dialog}
\index{Tab page creation}
\code{ //Taken from CodeBeautifiers Plus.\\
private const string TabAStyle = "AStyle";\\
private const string TabJcf = "JCF";\\
private const string TabCBGeneral = "Beautifiers";\\
protected override void IDERegisterTabs() \{\\
\tab base.IDERegisterTabs();\\
\tab RegisterTab(CreateTabNode(TabCBGeneral, 
typeof(Gui.CodeBeautifiersGeneralPage)));\\
\tab RegisterTab(CreateTabNode(TabCBGeneral, TabAStyle, 
typeof(Gui.AStylePage)));\\
\tab RegisterTab(CreateTabNode(TabCBGeneral, TabJcf, typeof(Gui.JcfPage)));\\
\}\\
}

It is common to provide a form that allows users to modified preferences. You 
can either register a tab page in LeXDK FormPreferences or do such a form 
yourself.

If you decide to add a tab page to FormPreferences, you can follow the above 
sample.

In the sample, totally three tab pages are added. You can see two type of tab 
pages here:

\begin{description}
  \item[No Parent Tab Page] LeXDK puts this type of tab 
pages under the ''Features'' item . For example, ''Beautifiers'' is in this 
type.
  
  \item[Tab Page with Parent] Other two tab pages in the sample are in this 
  type. Their parent is ''Beautifiers''.
\end{description}
  
You should also add a corresponding tab page type as the second or third 
parameter (omit this Type parameter if this tab node is without a page). Inside 
LeXDK, the page instance is delay-initialized to speed up loading process.

The tab page classes must be subclasses of CustomPage class in LeXDK, which is
of type User Control in fact. Only in this way you can use Form Designer to
visually design the page, which is much easier than old LeXDK methods.
\boxing{You needn't override IDERegisterComponents now, and it is removed from CustomFeature class.}

At last, you must register the tab page you define to the feature's 
preferences. PreferencesToUI and UIToPreferneces functions should be 
overridden in this case. \code{

public override void PreferencesToUI( ) \{\\
\tab base.PreferencesToUI();\\
\tab switch ((ParentType)PropertyService.Get("ParentType", ParentType.Default)) 
\{\\
\tab \tab case ParentType.Tools:\\
\tab \tab \tab rbUnderTools.Checked = true;\\
\tab \tab \tab break;\\
\tab \tab case ParentType.CNPack:\\
\tab \tab \tab rbUnderCnPack.Checked = true;\\
\tab \tab \tab break;\\
\tab \tab default:\\
\tab \tab \tab rbNewMenu.Checked = true;\\
\tab \tab \tab break;\\
\tab \}\\
\}\\

public override void UIToPreferences( ) \{\\
\tab base.UIToPreferences();\\
\tab PropertyService.Set("ParentType", GetCurrentParentType());\\
\}\\
}

\subsection{Without Registering Global Preferences Dialog}
You may be happier to have a dialog of your own which allows users to configure
the features. It can be done. Make such a configuration dialog/form in your
Plus, and define a configuration menu item in your feature which creates the
dialog for the user.

Also, there are other ways. If you take a look at ArtCSB Plus, you will see how 
ArtCSB code allows the user to change options from the Designer Navigator Form 
directly.

\section{About Assembly Info}
For your plus, I suggest the following items should be 
contained.\index{Assembly info}

\code{ using System.Reflection;\\
using System.Runtime.InteropServices;\\
using System;\\
using System.Runtime.CompilerServices;\\

[assembly: AssemblyCulture("")]\\

[assembly: ComVisible(false)]\\

[assembly: CLSCompliant(true)]\\

[assembly: AssemblyTitle("*")]\\

[assembly: AssemblyDescription("A Plus for CBC.$\backslash$r$\backslash$n" + 
"Codename: *$\backslash$r$\backslash$n" + "Written by *.")]\\

[assembly: AssemblyConfiguration("*")]\\

[assembly: AssemblyVersion("*.*.*.*")]\\

[assembly: AssemblyProduct("***")]\\

[assembly: AssemblyCompany("*")]\\

[assembly: AssemblyCopyright("All Right Reserved. Copyright (C) ****-****, 
*")]\\

[assembly: AssemblyTrademark("*")]\\
}

Generally speaking, Culture should be left empty. Title is used as a short name 
for the plus while Product indicates the full name. Description contains simple 
words you make for the plus. Full description for each feature should be written
in its .plus2 file instead.

Configuration is used to indicate the license covering the plus code. You can 
use any proper licenses. 

Version is specific for the plus build.

Others can be filled as you like.

The above information you provide for the plus is used in Plus Manager feature 
containing in Utilities Plus.

\section{More...}
Remember that this is an SDK for .NET OTA, any .NET languages can be used and 
mix-language programming is allowed. For example, AddMany 4.1 is totally in 
Delphi for .NET with a simple wrapper written in C\#.

The SDK itself can be extended by adding more classes assemblies, or called 
Minus (which reduces OTA complexity and provides more basic functions and APIs).

You can even extend or reimplement Framework and 
Lextm.Code\-Beauti\-fier\-Collection.Main in the framework using LeXtudio 
Wrapping Technology.

Although I choose GPL/LGPL in order to keep the core open source for ever, you
can still write pluses or minuses (even not open source) to meet your needs.

This SDK now is still in its early age. Many interesting features or libraries 
I write are not mature enough to be added in. So keep waiting, or take part in. 
If you write a Minus or Plus, send it to me (binary or source code). I can 
publish the news on the homepage of CBC so that others know your creation.

I have to announce that in the beginning stage of LeXDK, many names can be 
changed frequently. So, always refer to the newest Guide, and old versions
might be out-of-date.

Let us make a community of our own.
