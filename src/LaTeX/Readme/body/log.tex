\chapter{Change Log}
This is different from the history chapter in the manual since more details are
provided. I provide them just in case you meet same old problems one day. Wish
they can be helpful.

It should be noted that the date of announcements are usually a little bit
earlier than the date of publishing (on BDN Code Central) because tests of the
binaries take some time.

There are also some unreleased versions here. Since NixNewNer, nearly all
versions (including RCs) can be downloaded from CBC's GForge homepage.

\begin{itemize}
  \item 2008-02-08 GrapeVine Update 1
\begin{description}
\item[RC1]
\begin{quotation}
Now File Aid supports Delphi VCL projects. 
\end{quotation}
\end{description}
  
  \item 2007-10-22 HardQuery Release 2 Update 4
  \begin{description}
\item[RC1]
\begin{quotation}
Bug fixes from GrapeVine M8 is ported back to HardQuery Release 2. It is good
news that I can port back this fixes. However, I cannot setup a build
environment easily so this version is released as source code only.
\end{quotation}
  \end{description}
	\item 2007-07-07 GrapeVine
	\begin{description}
\item[Release To User]
\begin{quotation}
Because of a bug, JCF 2.34 is not enough for GrapeVine Final. A customized
build is made with help of Anthony Steele
\end{quotation}
\item[Milestone 11]
\begin{quotation}
You can see from this log that I started GrapeVine in early July. The funny
thing is that I spent nearly five months on this release. It is not because I
did not want to deliver it earlier but due to my job.

I do not have lots of time now to work on this open source project, so
general progress is slow. After learning from Paint.NET, I decide to release
often. Thus I divided the process into so many milestones, and after about two
weeks I released a new build.

If you check Google Code homepage for Code Beautifier Collection you can see
this site is much more popular than old site on GForge (6.0 binaries download
count reaches over 2,500 in 5 months).

6.0 Final is the first Final version on this new site, so wish all of you enjoy
this release. I made a few changes from M10. First, I used Vista task dialog to
replace several customized dialogs I designed. Task dialog is an important
creation in Windows Vista which standardizes GUI elements to acquire user
input. Second, I tried to make InDate process smoother with a hidden bug fixed.
I changed a few captions to better describe the steps.

JCF 2.33 is now included.
\end{quotation}
\item[Milestone 10]
\begin{quotation}
Have to turn off the experimental installer hack for ''install for all users''
now. It will be available only if I think of a better idea.

Also I got a report of issue 6. I forgot to check files included in the
installer last time. So jcf.exe and astyle.exe were missing. To prevent it from
happening again, I removed all '*' from Inno Setup script so the compiler will
check if any file is missing for me automatically. Thanks to this nice feature
of Inno Setup compiler. And I stopped to use Inno Setup build-in Post Install
actions to run installforallusers.exe. I rather run it in [Code] section
manually.

JCF 2.33 is not included in this build because it contains an ASM code
formatter now which is an important update across versions. I will handle this
change later because now I need to maintain a separate project JCFStyle for
GrapeVine which most makes use of JCF source code directly.
\end{quotation}
\item[Milestone 9]
\begin{quotation}
A CEO reported an issue to me that 'Install for all users' was not correctly
implemented. So I decided to ask for help on CodeGear newsgroup. And it seemed
that this issue cannot be fixed easily. That's why I still mark this feature
'experimental' in this build. I have done a few tests about the changes I made,
such as hacking bds.exe. It works on my machine so I wish it runs on others.
But you can always expect issue report some time. I'd like to overcome the
original issue but I may have introduced more (because of the way I hack).

Also I have added AddMany Plus again but without enough testing. So please help
test this feature, my dear users.
\end{quotation}
\item[Milestone 8]
\begin{quotation}
I keep a long list of features, and when all of them are done, I can say that
Code Beautifier Collection 3 is ready (current version is still 2.6.0.0).
Compared to JCF which is now 2.31 (31st build of version 2), I can predict that
I have a long way to go.

In earliest versions of User Manual, the TODO section is an important part, but
some items are hard to implement. JCF Style Editor is one of them which is
also a critical part of CodeBeautifiers Plus. And eventually I spend last
weekend on this item and got it done. The approach is rather easy, just make
use of JCF source code completely if possible. I haven't touched Object Pascal
code for a few months, so it is not that easy to get back to the games at
first. Then I ripped out most forms of JCFNotepad project. When only elements
related to fAllSettings form left, I wrote a simple dpr to launch this form and
everything is just okay. Wonderful.

I'd like to say I love Delphi 2007 as well as RAD Studio 2007 which makes full
use of MSBuild that simplify both Delphi/C++Builder project structure a lot.
For example, my new JcfStyle project only needs one dproj project file (no
bdsproj, cfg, or dof). Forget to mention that dpr is in fact a Pascal
source file which I suddenly remembered.

The JCF style editor is the most important new feature in M8. I am sure that
you will like this new stuff. Stay tuned.

Also a newly found issue 4 is resolved in this version.

When uninstalling CBC, there was an exception. Yes, I had to work around Inno
Setup implementation here in order to make installforallusers.exe runs well.
Luckily, this issue is resolved in this build.
\end{quotation}
\item[Milestone 7]
\begin{quotation}
Issue 3 on Google Code is resolved in this build. Also I update the Inno Setup
script and MSBuild script. In this way I can ensure the installer is correctly
built. This is also a bug fix version.
\end{quotation}
	\item[Milestone 6]
	\begin{quotation}
	A big issue is reported on CBC Google Code Homepage. M6 is aiming to fix this
	issue, so is only a bug fix version.

	Also after a basic test of installer on test machine, I found and fixed more bugs.
	One of them prevents InDate from downloading unstable builds, while another
	prevents new version from installing.

	Have to say a .NET assembly is much different from Win32 DLL. You must specify its
	AssemblyFileVersion in code so Inno Setup Preprocessor can read its file version
	correctly.
	\end{quotation}
\item[Milestone 5]
\begin{quotation}
CodeGear has done a great job to release Delphi 2007 Update 3 with RAD Studio 2007
so soon. And it also requires a new version of GrapeVine. That is why I release this
M5 now. No new features is added now.
\end{quotation}
\item[Milestone 4]
\begin{quotation}
I planed to provide more Vista related UI elements but I failed. Beside M3 failed
to launch with RAD Studio, e.g. Delphi 2007 Update 2. I had to cut off a few
items from list and focus on some important things, such as compatibility and
stability.

That's why I release this version now without much improvement. Yes, wish I
could do more in M5.
\end{quotation}
	\item[Milestone 3]
	\begin{quotation}
	UAC support is complete in this version because Visual C\# 2008 Express Beta
	2 is used to compile the projects.

	The installer is redone in order to make ''install for all users'' possible.
	Also many changes are introduced to folder structure.
	\end{quotation}
	\item[Milestone 2]
	\begin{quotation}
	This version starts to include Vista API library in order to provide better UI
	on Windows Vista. UAC compliant is experimental.
	\end{quotation}
	\item[Milestone 1]
	\begin{quotation}
	In this version, I just compile against .NET 2.0. As a result,
	TBackground\-Worker component is removed. Also no ActiveX component is used.
	Visibles.dll is disassembled to C\#. In order to work around OTA changes, some
	features are removed or reimplemented. The version number is updated to 6.0.
	\end{quotation}
	\end{description}
  \item 2007-03-07 HardQuery Release 2 Update 3
  \begin{description}
	\item[Release To User]
	\begin{quotation}
	Actually RC 5 is finally released to users as the final version.
	\end{quotation}
    \item[RC 5]
    \begin{quotation}
          A few days ago I found that if I tried to use Typing Speeder while editing
          TriggerKey.cs unit unhandled exception occurred. I tried to locate the
          bug and found the Code DOM section in OTAUtils unit hard to read and
          understand. As a result I started to update the code.

          Today I have done it all. A new namespace ''LeXDK.Co\-de\-Dom'' is added.
          All Code DOM related stuffs are moved there. Also the bug mentioned
          above is solved, too. Yes, the old version cannot handle delegate
          types well and the new version can. That is how now Source Navigator
          and Typing Speeder are fixed. I find another OTA bug. Although the
          delegate type can be processed correctly, the line number cannot be
          located like the constructor case.

          Since I have been prepared to leave for Shanghai, the project will be
          silent soon. Wish I can deliver the final Update 3 soon. Yes, I try
          to add some Delphi 2007 support but I do not have a copy to try.
    \end{quotation}
    \item[RC 4]
    \begin{quotation}
          I just added something really useful in BagPlug's Typing Speeder right
          now. Yes, it was something I've been dreaming of for a long time.

          Whenever you want to add some XML comments in C\#, you have to type a lot
          of words and tags. SharpDevelop provides something really cool. If you
          type ''///'' before a line of declarartion, the comments for that item
          will be generated for you. That is why I use SD a lot. However, BDS
          does not have a similar feature. And it is what I have done now. Now
          Speeder will help you when you type ''///'' and press a Space key
          then. It will generate most of the comments for you using CodeDOM. For
          that reason, a DOM bug prevents it from doing something for
          constructors.

          I thought it would be easy to extend this to Delphi files. However, I
          found that the Delphi CodeDOM was really different. The line number
          for a method is not the first line but the first line after ''begin''.
          That is why Delphi support is not available.

          In fact Chua Chee Wee's Productivity Experts 0.8 (which is a Code
          Central entry \#23380) inspires me a lot. However, we use different
          approaches. Chee Wee reads a few more lines after ''///'' and parses them
          using regular expressions while I am so lazy that I use CodeDOM
          directly. It is lucky that I have done Source Navigator ahead, and a
          lot experience can be applied here.

          A few LeXDK lines are modified because of tuning. Get\-CurrentLine is
          changed to GetLineOf, which is much faster. GetTotalLines is also
          tuned.

          Also I found a OTA bug. Yes, IOTAEditPosition.Read function can
          make thing hard. When I count the char number by columns the TAB
          char is counted as 4 chars, so when reading the following line will be
          touched. A workaround is provided here.
    \end{quotation}
    \item[RC 3]
    \begin{quotation}
          A new feature of Icon Browser is added now. It gives you access to an
          open source utility named Icon Browser made by Kenny Kerr. Also I find
          that Path.Combine is a well function.
    \end{quotation}

	\item[RC 2] \
	\begin{quotation}
        Generally speaking nothing is new. However, most efforts were placed
        on the moving from NAnt to MSBuild. The work is very funny. MSBuild is
        not very similar to NAnt but most knowledge is still applicable. The
        changes were made whenever I could not simply translate a NAnt line to
        MSBuild. Significant ones are: 1) TipOfTheDay project is now merged into
        Utilities Plus. 2) Install and Templates projects are now part of the
        solution. 3) .plus2 files are now placed in the root folder.

        I never thought I would create my own .targets file so soon. In order to
        simplify the build of TeX files, a Lextm.\-LaTeX.\-Tasks.\-targets was
        written. Also MSBuild Community Tasks were used to zip up files. This
        makes my building easier but users may spend more time on setting up the
        build environment.

        MSBee tasks cannot be perfectly called from within a .proj file, so I
        have to use an Exec task to call MSBuild outside the main process. The
        same problem happens when I want to package source files. The evaluation
        of wildcasts is so early that I have to use a Exec task again to ensure
        all generated items are included. I think there is alternative way and I
        will see if I can find it soon.

        The line counter feature now has basic Delphi code counting support.
        However, after reading the source code I think the algorithm is still
        buggy, so the feature is really experimental. The parsing of a
        real-world source file should be much more complicated.

        The loading problem found in RC 1 is now fixed and disabled plus
        assembly will not be loaded into memory any more.

        By the way, this version is now still named a HardQuery version while in
        fact a lot of changes made here were scheduled for GrapeVine. So I
        should better call it the first GrapeVine version to go along with the
        changes. Yes, so many changes.
    \end{quotation}

  \item[RC 1] \
  \begin{quotation}
      InDate is modified again. This time GForge turned back to HTTP. I had been
      thinking of switching to Google Code, but since I could not upload
      packages there the moving would be delayed. The change in InDate this time
      will surely make the migration smoother.

      I have been making a few features of my own in the latest versions of CBC,
      such as Readme Notifier, Typing Speeder, and Source Navigator. Because I
      wanna keep my code GPL/LGPL covered, I finally extract them from
      Wise\-Editor and move to a new BagPlug Plus. So in fact nothing is really
      new here.

      I rename a few features to get along with the manual. I will try to
      update the pictures, too. Besides I do a little clean up of the code.

	  Although all features of a plus are disabled, when launching, the plus is
	  still loaded into the memory. It is a big bug. I will fix it soon.

	  A new feature is added named Line Counter (original named GroupStats). The
	  initial version was based on King's Tools for Visual Studio. However, the
	  implementation was soon replaced by a port of LineCounter feature from Jon
	  Rista and Daniel Grunwald's Add ins for Visual Studio and SharpDevelop.
	  Although Delphi OTA is different from VS or SD's API, the porting is easy.
	  The problem is that Delphi code cannot be counted now. It should be easy to
	  extend the code, so I wish the function can be added soon.
\end{quotation}

\end{description}
  \item 2007-01-14 HardQuery Release 2 Update 2
  \begin{description}

    \item[Release to Users] \
    \begin{quotation}
      Since a national holiday, the Chinese New Year, is coming, I think I don't have
      enough time to add more to the code tree, so will stop here. Finally I am near
      the end of signing a wonderful contract to start my career. Wish all of you
      happiness in the new year, the Pig year.

      Even if AutoUpdate is set, InDate'd better check for updates once a day. This
      function is added from now on.

      Also when you open a file of a project, such as a .build file, because it are
      not a ''source file'', the container project will not be activated by CBC in
      the older versions. It is only in this version, the project can be activated
      for you.
    \end{quotation}
    \item[RC 5] \
    \begin{quotation}
      A small change to the Doc Preview form makes all links inside disabled. Only in
      this way the old unwanted action of navigating to invalid links is handled.
    \end{quotation}
    \item[RC 4] \
    \begin{quotation}
      AutoComplete feature has been heavily updated in this version, so many
      hard-to-read code now is clearer. Null Object pattern is used twice here to
      ensure simplicity.
    \end{quotation}

    \item[RC 3] \
    \begin{quotation}
      Matt Berther created a unit AutoCompleteComboBox. Although it is not perfect, I
      find it useful in Source Navigator. As a result of using it, it is easier to
      use the UI now.

      Also I refactor AutoCompleteFeature again and create a new namespace for it.
      Registry and Command patterns are used in order to clear the code. I leave a
      few bad smells there because I am not ready to remove them this time. I will
      see to them in later iterations.

      After some more digging I will release RC 3 as the final RC for Update 2.
      \#39 is such a big bug that I feel that a new and stable version should be
      coming quickly enough this time.

      At last I wanna comment on two AutoComplete\-Combo\-Box available on The Code
      Project. In fact they achieve similar functionality. However, when the text
      is auto-com\-ple\-ted, the list is not drop down so you are not 100\% sure the
      choice is correct. A better solution may be dropping down the list and
      highlighting the choice but the user can see similar choices in the list (the
      list can be filered). This is a kind of UI David used in File Aid. Yes, I
      like this kind of UI and I want to make such a custom ComboBox that is
      suitable here in SourceNavigator.

      Now after reading .NET 2.0 SDK Documentation, I see that ComboBox there
      already supports auto complete feature. After a few tests, I find it good
      enough. As a result, I will update Source Navigator later.

      Soon I will release the final version.

      As a matter of fact, this time I make big changes in Auto\-Complete feature.
      For example, ''else if'' is no longer a keyword, because in this version only
      one word symbols can be keywords. If ''else if'' should be supported, the
      architecture needs to be modified. The verification of a keyword must be more
      flexible. KeywordCommand.Verify(IOTASource\-Editor) would be okay. But if
      foreach is used to iterate a long list, the efficiency will be low, and it
      will be hard to tune the process.
    \end{quotation}
    \item[RC 2] \
    \begin{quotation}
      Today I received a bug report about \#39, and I blog about it
      here\hyperlink{http://blog.csdn.net/lextm/}
      \hyperlink{archive/2007/01/17/1486044.aspx}

      This is a serious bug so I decided to fix it ASAP. I am glad that it has been
      reported and fixed now.
    \end{quotation}
    \item[RC 1] \
    \begin{quotation}
      I am working on fixing bugs and upgrading certain features from SBT in this
      version. The first step is on the File Aid feature. At first, because BDS 2006
      IDE manages the tabs of Editor fine, I do not want to keep the Open Tabs tab
      page any longer which complicates the code.

      Also, I choose File Aid because this time more UI elements will be
      processed and updated (I have to admit in the early versions I paid not
      anough attention to them, so many elements behave unconvenient when you
      resize the forms). I will focus on this kind of problems this time. The
      tab order problems will be solved in a later version. As a result, I need
      to find all the forms and user controls quickly. I turn to File Aid, but
      it works bad. Because of a OTA bug, not WinForms are listed in the dialog.

      After a lot of tests, I use maybe the simplest way --- a form or user
      control file a.cs must have an a.resx with it. I know there are some
      special cases but they are rare. Since it is the fastest method, I prefer
      it and use it at last.
    \end{quotation}




  \end{description}
  \item 2006-12-28 HardQuery Release 2 Update 1 Hotfix 1
  \begin{quotation}
    I have to confess GForge makes me angry this time. For a long period, I have
    been sad that the service is vulnerable. Also you can only use one thread to
    download the packages.

    Lately, GForge has used a lot of HTTPS instead of HTTP. As a result, InDate
    cannot download anything directly from the server and cannot work correctly.

    Luckily this time I have to switch to other service or drop InDate for months,
    and only a few changes in InDate make everything okay. Since HTTPS prevents
    InDate from downloading the update list and packages, I place the list on my
    blog instead. When InDate find an update package, it will send the link to your
    default internet browser (Firefox or IE) instead of downloading directly. In
    this way, you can still receive latest information on updates. As a result of
    so many changes, you receive this Hotfix today. I write a readme file to tell
    you how to install it besides. In fact, I find the changes rather slighter than
    I ever imagined. What I have done is a few if-elses and a few new
    PropertyRegistry items. Thus, once I need to turn back to the old InDate
    functions, all I have to do is changing some properties in the settings. Oh, it
    is rather easy this way.
  \end{quotation}

  \item 2006-11-11 HardQuery Release 2 Update 1
  \begin{description}
    \item[Release To Users] \
    \begin{quotation}
      It is already December and 5.3.1.1003 is stable enough, so I decide to
      release the final version 5.3.1.1123 soon. I am sure it will be published
      before the X'mas because I want it to be a present that I send you all my
      dear users and friends.

      Finally I use log4net in CBC to do the logging tasks. I knew this
      library months ago and tried to learn it but I could not find a
      convenient way to adopt it. I have to say Marc Clifton's Debug unit is
      such a clean and easy way suitable for most cases that I can hardly
      drop in a minute. At last, when I need more powerful features like
      different logging levels which it cannot provide, I find a good reason
      to leave it aside.

      Also it is not easy to make use of log4net because it lacks certain
      features such as Indent and Unindent. Compared to CodeSite, a famous
      logging and debugging tool in Delphi field, log4net is not mature
      enough and need further improvement. I am glad that this time I have
      the power to make something useful myself. If you read the
      LoggingService unit, you will find I make a wrapper for log4net which
      adds a few interesting and common functions such as Indent, Unindent,
      Enter, and Leave. In this way you can use log4net in a way similar to
      plain old Debug and Trace. Because I find this unit very useful I
      place it in Lextm.Common assembly and under LGPL, in that way you can
      use it anywhere you like.

      In the last minutes I have found ways to get parameter information out
      of CodeDom classes. As a result, now when Source Navigator shows the
      method list, parameters are listed, too. Thus you can distinguise
      overloading functions easier. The limitation now is that I only
      implement the C\# version, so even functions in Delphi shows in the
      C\# way. I will further improve this feature in later versions.

      Finally I remove two boring limitations in feature development.
      Remember you have to make the feature classes Singleton? Now it is not
      required. A newly-added unit FeatureRegistry will ensure only one
      feature instance of one type exists. What you have to make sure is
      that the feature class must have a public constructor. Wish this could
      reduce your work on plus development. The only change is that the
      order of menu items is totally different from early versions.

      Also I find out the AutoFlush feature of PropertyRegistry is not
      consuming too much time. As a result, I will keep AutoFlush field true
      because manual flushing is prone to errors. Now Flush is only done
      when AutoFlush is false. In this way, one day I can turn AutoFlush to
      false without fearing something goes wrong somewhere.

      Also I change the final version number to 5.3.1.1124.

      Just right now I know how David changes the names of AntMenus at
      run-time. Yes, just assign new strings to the menu items. As a result,
      I know also how Marc Rohloff implements his FavouritesMenu. Maybe one
      day I will drop the implementation of a Favorites Dialog and try
      Marc's way. But both the approaches have ups and downs.

      WiseEditor is now 1.0.1 and NFamily is now 0.9.1.

      The documents are updated again slightly this time.
    \end{quotation}
    \item[RC 3] \
    \begin{quotation}
      In this build, I replace the former InDate UI with a compact one. After
      that I add such an option that whenever BDS is started, InDate checks for
      available updates.

      Also I port Martin R. Gagn$\acute{e}$'s loading circle control back to
      .NET 1.1, and use it in InDate UI. Since it is hard to predict the
      progress, I do not prefer a progress bar control. This loading circle is
      more suitable.

      I spend time on learning Microsoft shared source Enterprise Library's
      Logging Application Block for .NET. Yes, it is well designed and easy to
      use. Similarly, I find I can use log4net quite well now. But both of them
      require me to add something to bds.exe.config - because CBC is not an
      executable, and is loaded by bds.exe. You know it is not hard to do it on
      my machine but may be trouble for others. Meanwhile, it requires great
      effort to transfer from using Debug and Trace to either approach above.
      As a result, I will keep on using the old method for a few month.
      However, if I ever start a new project, I must pay attention to log4net
      or Enterprise Library because they provide more features and greater
      power to log down information.

      When I try to add something to bds.exe.config, I find that I can change
      the runtime of the IDE to .NET 2.0. Also, this is possible. Although I
      can even let C\# compiler 8.0 (.NET 2.0) compile the project. The problem
      is that the WinForms designer is now .NET 2.0, the auto-generated code is
      not .NET 1.1 compatible. I do not know how to change other parts of the
      IDE and add references to .NET 2.0 assemblies. I have to stop my
      experiment later because Highlander will surely solve all the problems.

      Now I have found a method to use log4net in CBC. Yes, it is not hard to
      get used to this kind of logging system. However, it requires a lot of
      modifications of code. So I will finish the transformation later and find
      a way to prevent logging for release versions. That will not be hard I
      think.
    \end{quotation}
    \item[RC 2] \
    \begin{quotation}
      In RC 2, I add an easter egg with the help of Sharp Dev Tools. Open the
      About form and use Alt + TEAM, and then you will see it. Yes, that is a
      picture of David ''The Creator'', or David Hervieux --- father of Sharp
      Builder Tools.

      When designing this feature, I find that SDT is badly documented.
      Something you have to do is not listed somewhere. For example, it is easy
      to follow the demo to initialize and set SecretKeyManager, but you may
      forget to set the Form's KeyPreview property to true. I will soon make
      use of other classes in SDT so I will write more tips about using it
      later.

      Now MSBuild is used when building projects. However, I still use NAnt to
      clean up folders and call other targets. Why not turning to MSBuild just
      in one step? First, there is a lot of work ahead and I am not sure it
      will be easy. Also in certain aspects NAnt is not bad. As a result, I use
      MSBuild to reduce duplication. When I was using NAnt, I had to keep SD
      projects while keeping NAnt scripts about project building. Now only SD
      projects are required to be kept. The only problem of this change is that
      you cannot specify which version of ToolsAPI assembly should be linked
      because those SD projects are for BDS 4 exclusively.

      There is something I have to add here about MSBuild. When it is shipped
      in .NET 2.0, I know its existence. Sharp\-Develop uses its own targets to
      override Microsoft's targets so as to make the projects .NET 1.1
      compilable. However, there is a big gap between .NET 1.x and 2.0, so SD's
      workaround is not perfect. Although the projects can be compiled against
      .NET 1.1, the resources files are not linked into the final assemblies.
      As a result, every forms that contain images triggers exception when it
      shows. How to work around this? Microsoft provides MSBee, a set of .NET
      1.1 only targets. Now I use MSBee and everything works fine except one -
      I cannot let you build CBC in SD 2.1 now (if you do compile in SD 2.1,
      the assemblies are against .NET 2.0 so BDS cannot load them). Instead,
      you should use MSBuild command line to achieve this goal. At this moment,
      I cannot find a better solution, so stay tuned.

      JCF 2.25 is bundled now.
    \end{quotation}

    \item[RC 1] \
    \begin{quotation}
      Since NDoc is no longer maintained, everybody who have used it is searching for
      an alternative way. As a result, Sandcastle is now dominant in this field even
      though it is still in CTP age.

      I did not know this tool until I read the roadmap of SharpDevelop. Yap,
      SD will rip NAnt and NDoc in the future and use MSBuild and Sandcastle
      instead. I am switching a little bit faster. Now, LeXDK reference file is
      generated by Sandcastle (with a lot of help from Sandcastle Help File
      Builder. Thanks a lot, Eric Woodruff) in Visual Studio 2005 help style
      which is more beautiful than before. I have updated the NAnt script. In
      order to catch up with the trend, I will soon drop NAnt and turn to
      MSBuild.

      Also two bugs are fixed in this version. One is an unexpected exception
      in the C/C++/C\# tab, and the other is deadlock in WiseEditor.

      Updated content includes JCF 2.24, and a new version of deep serializer
      from Marcus Deecke.
    \end{quotation}

 \end{description}

  \item 2006-10-21 HardQuery Release 2
  \begin{quotation}
    One more feature is added in this version. When you work in a project group,
    the project that contains the current active file in the editor will be
    activated automatically.

    All documents are updated to meet changes in LeXDK 5.3.
  \end{quotation}
  \item 2006-10-17 HardQuery Release 2 Milestone 5
  \begin{quotation}
    Since no bugs are there to fix and much more features are added, I drop
    the plan of Update 1 and announce Release 2 of HardQuery.

    Many new items are now included inside LeXDK. The newly-added
    PropertyRegistry is a convenient way to register an option or a
    preference (now they are refered as ''properties''). As a result, LeXDK
    5.3 stops providing you preference service, which was introduced in LeXDK
    5.2.

    Deep serializer and deserializer are added in Milestone 5. Both reduce
    development complexity.
  \end{quotation}

  \item 2006-10-11 HardQuery Update 1 RC 4
  \begin{quotation}
    Minus bug in InDate is resolved. Also an Elide to definition menu item is
    added under Edit.

    Because UnhandledExceptionManager is used, every unexpected exceptions can
    be catched now.
  \end{quotation}
  \item 2006-10-03 HardQuery Update 1 RC 3
  \begin{quotation}
    When David released SBT 3.1 final, he droped OTAMenuEnhancement and used
    OTAEditorEnhancement instead. However, when I ported those OTAs into CBC, I
    chose the dropped unit. What a mistake! Using BeyondCompare or WinMerge to
    compare the two units you will see many important differences.

    Luckily I notice this now and make use of EditorEnhancement unit
    extensively. I do something beyond. More functions are added to
    BaseLanguageCodeHelper unit in order to make use of the factory everywhere
    needed. CodeBeautifiers Plus now has a few units removed. They are now
    inside the SBT Minus. It is a change that I have been waiting for, because
    now tools such as AStyle and Jcf is decoupled from CodeBeautifiers unit
    finally. Further changes may make them completely independent, but I am not
    sure.

    Since there are big changes, I increase the version number again.

    C\#Goodies's ConvertXml class is dropped finally. I write a class for my
    personal use and it is much simpler to read and understand. Also, it is
    inspired by Marco Cant$\grave{u}$'s Mastering Delphi 2005. Thanks Marco.
    You are a great teacher.

    How to open shell at any location? I have to run cmd.exe there. It is not
    hard to implement, but isn't easy to find it out. David used ''/k cd
    $<$path$>$'' which does not run properly on Windows XP. So now I provide a
    work around, that is running cmd.exe in the folder.

    I learn CodeDOM using SBT and ArtCSB's source code. Yes, it is an
    interesting technique. Although the implementation in .NET 1.1 prevents me
    from getting enough information about class's constructor, I make an
    easy-to-use Source Navigation Bar which is common in Visual Studio and
    SharpDevelop but rare in Delphi. Castalia has such a feature, which is
    buggy because of a not-good-enough C\# parser. I use .NET build-in DOM
    which is also buggy. SharpDevelop has a better parser but I do not keep its
    1.1 version source code. Also, I think in GrapeVine I can make a better use
    of SharpDevelop 2.0/2.1 source. David implemented something like
    ViewCodeDomForm, but he chose a strange UI. I prefer a bar to a tabbed
    form. That is also why I do not use ArtCSB Source Navigator finally. A tree
    form is not my choice. This is the most important addition to HardQuery in
    my opinions. It is quite useful and easier to use than Structure Panel of
    BDS. Currently, this feature only supports Delphi (both Win32 and .NET) and
    C\#. The limitation is that only CodeDOM for the two are realized in the
    IDE OTA.

    Two open source tools, Clearer and ProcessChecker, are added into the
    bundled folder.
  \end{quotation}
  \item 2006-10-01 HardQuery Update 1 RC 2
  \begin{quotation}
    Rest of SBT features are ported finally in this version. I think most of
    them are useful and tiny, so migration is not too hard. Except the
    Singleton limitation, there is something boring about CustomFeatureTool.
    One feature can only have one tool class related. So even two tools who are
    similar should belong to two distinct features. It is not convenient, too.

    Most of the newly-added features are tested. They works fine. So, I have to
    arrange a date to release the update final.

    Now it is even longer for CBC to load all its features. It is hard to tune
    the performance now because I am lack of techniques. I will read Effective
    C\# again to see what I can do then. Also, you can turn off the features
    you do not need to reduce the loading burden.

    I am planning to divide Framework and WiseEditor Plus into a few smaller
    pluses. However, the process will end until GrapeVine.
  \end{quotation}

  \item 2006-09-28 HardQuery Update 1 RC 1
  \begin{quotation}
    All ArtCSB 2.5 features are imported in this version because I think an
    auto-saver can decrease the risk of losing all changes if system crashes.
    Also, fold to definition is friendly. I ported Source Navigator in order to
    know more about CodeDOM.

    It is done without difficulty. Why? I have to mention once again, LeXDK 5.2
    is much better than before. So any migration is easier. However, there are
    still problems. Why are all features Singletons? It is not convenient.
  \end{quotation}

  \item 2006-09-16 HardQuery Final
  \begin{quotation}
    I am glad to announce the final version here after so many testings and
    debugging. It is two month after the first milestone version of HardQuery.
    The process is so long that many users may find NixNewNer Update 1 buggy
    enough. Yes, many bugs are found during this period either by me or by some
    kind-hearted users. I cannot promise this final version is bug-free, but I
    am sure it has much fewer bugs.

    M12 is stable enough except it lacks some BASIC language support. I have
    no time to tune the performance this time because hard work has been done
    in the summer. Problems that have not been addressed will be considered
    in GrapeVine.
  \end{quotation}

  \item 2006-09-13 HardQuery M12
  \begin{quotation}
    I am promoting the Turbos these days. My very first target is a friend
    who teaches in the local high school. Luckily, she is teaching Pascal now
    and using Turbo Pascal already. When I help her install Turbo Delphi for
    Win32, she soon realises that it is a brand new and so modern IDE. The
    only problem now is that I am not sure whether Turbo Pascal projects run
    well in Delphi IDE. I will try it.

    With the help of the SharpDevelop Team, AStyle for Sharp\-Develop is now
    in Beta 3, which is much closer to a Final version. What I have known
    from this process is that BDS OTA is much more complicated but
    sometimes easier to extend.

    M12 is mainly focused on bug fixes. \#32 and \#33 are fixed. They are
    really big bugs I see.

    How did I find them? It was so lucky that when I taught my dear friend
    to use Turbo Delphi, I was using BDS Delphi Personality to create a
    Console Application. I failed while she succeeded. Why? The newly added
    Readme feature had a bug that only occurred in this case. I want to
    point out that it is more likely a BDS OTA bug/limitation. Creating
    applications in different personalities seem to be completely different.

    The Plus Manager bug was found later when I tried to locate the Readme
    bug. I wanted to disable fully-tested features in order to isolate the
    bug. Then I found PM failed. In fact, currently PM code has a lot of
    bad smells. I will redo this feature in GrapeVine. I have a good plan
    now but I cannot make it under .NET 1.1.

    More tests will be taken recenly. I wish to deliver a stable final
    version.
  \end{quotation}

  \item 2006-09-10 HardQuery M11
  \begin{quotation}
    Why it is fast named as M11 just one day later? There is a big
    improvement on CodeBeautifiers Plus, the UNDO support.

    If you read along this log you will know I had done an AStyle plugin
    for SharpDevelop. Since its Beta 1 cannot handle reloading after
    astyle.exe processes the file correctly, when I announce it on the
    SharpDevelop User Community, I ask for help.

    I never dreamed Daniel Grunwald would answer my question, who is a
    major developer of SD. He suggested me making a temporary file aside,
    formatting it and loading it directly to SD editor's buffer. As a
    result, this kind of formatting can enjoy SD's undo service.

    Today, I successfully finished the Beta 2 of that SD plugin. What's
    more? After analysing WinMerge feature's code, I found David
    manipulated BDS editor's buffer there. I went deeper into OTAUtils unit
    and found FillBufferWithFile and CreateFileFromBuffer right in the
    place.

    Although David did not leave a comment, you and I could understand the
    functions immediately. As a result, I redid something inside the CB
    Plus and everything was okay. However, the process was not quite easy.

    David's FillBufferWithFile implementation had bugs. After reading some
    OTA materials I found a solution after a few trials (poor documented
    OTA!!!).

    I was hurried to bring UNDO service in but I broke the old architecture
    inside CB Plus. Now, Project and Project Group menu items are disabled.
    I will make them work later. I do not know whether it will be in M11 or
    in a hotfix after final.
  \end{quotation}

  \item 2006-09-09 HardQuery M10
  \begin{quotation}
    I ported Liz's Readme Viewer from Delphi to C\#, because I found there
    are some problems to build a Delphi for .NET plus based on LeXDK
    (terrible). Now I guess something wrong with CLR compatibilty. Maybe I
    miss some settings. I will dig it later.

    This feature is included in WiseEditor Plus now. Whenever you open a
    project including a readme file (e.g. Readme.txt, Readme.rtf, and so
    on), the readme file will be opened automatically for you.

    Also, I updated the Inno Setup script to add ArtCSB Plus Beta inside.
  \end{quotation}

  \item 2006-09-07 HardQuery M9
  \begin{quotation}
    Since I ported ArtCSB Plus sucessfully, I decided to make another
    milestone that contained this feature.

    Also I found in ZetaLibNet a tri-state checked treeview control, which
    might be quite useful in next version of Plus Manager feature in
    Utilities Plus. However, that library is now under .NET 2.0. Although I
    could do something to bring it back to .NET 1.1, I would not. This kind
    of upgrades will all be available in GrapeVine versions.
  \end{quotation}

  \item 2006-08-30 CBC HardQuery M8
  \begin{quotation}
    M7 is released on GForge. Some tiny problems were found and fixed in
    that version.
    \begin{itemize}
      \item Timer unit is renamed to Stopwatch, which provides similar functions as
      Stopwatch class in .NET 2.0.
      \item Assembly unit is renamed to AssemblyHelper. Many functions are modified
      heavily.
    \end{itemize}

    I decided to release M7 as the last milestone for HardQuery, but it had
    some bugs, too. I found them and fixed them soon, so I provide M8 and
    wish it would be the last. Also, many beta features are disabled as I
    wish to do in the final. NFamily is now disabled by default.
    ComponentNamer in WiseEditor Plus is disabled, too.

  \end{quotation}

  \item 2006-08-19 CBC HardQuery M6
  \begin{quotation}
    LeXDK is refined to increase loading speed. Now preferences management
    is much easier because LeXDK provides that basic service. Feature names
    are removed, and preferences files extension is changed back to .ota.

    Some documents are updated. Last version is the last milestone,
    because LeXDK 5.2 is almost done.

    An important note is that beta version of features and pluses such as
    ComponentNamer and NFamily will be turn off when shipping. You can turn
    it on manually if you want to have a try.
  \end{quotation}

  \item 2006-08-12 CBC HardQuery M5
  \begin{quotation}
    I am quite glad that NFamily Plus Beta is done in this version. All
    features are successfully ported from SBT to CBC. Though a few bugs are
    there, I wish I could finish testing and fixing soon.

    Many things changes under the surface. For example. when you see
    FavoritesFeature class source code, you will find it is not as the
    same as most of other features because of its Preferences class.
    Also, ExpressoFeature shows a brand new way to load and save
    preferences with the help of newly-added PreferencesManager class.
    The changes reduce complexity of a feature dramatically. As a result,
    LeXDK DG will be updated and the word ''draft'' will be removed
    finally. In my opinions, LeXDK HardQuery (5.2) is the most stable and
    the most easy-to-use version since its birthday. Although later many
    tunings will increase efficiency, the architecture will keep
    untouched for a while.

    You may notice that I use M5 instead of RC5. That is because I change
    the release naming rules a little bit. Now every stage in Roadmap will
    be finally release as a ''Final'', such as HardQuery Final (it should
    be officially versioned 5.2.0.1123). Before a final version, there will
    be a few ''Milestone'' versions (I borrow this idea from the Eclipse
    Project). And after a final version, there will be ''Update'' versions
    which addresses urgent bugs or vulnerabilities. Between final and
    updates, there will be ''Update Release Candidate'' (RC) versions.
    ''Hotfixes'' will be released if needed.

    From now on, final versions are focused on adding features and updating
    LeXDK. Update versions and Hotfixes are focused on bugfixing and
    performance tuning. Milestone versions provide major advancements,
    while RC versions present immediate but unstable patches.

    Also, I wanna show something important about the new roadmap. The stage
    after is now named as GrapeVine (6.0). GV is a tough stage because it
    focuses mainly on HighLander and .NET 2.0. Many part of pluses will be
    changed in order to suit .NET 2.0. However, some changes such as
    FastInvoke are not available on .NET 1.x. In other words, GV only
    supports HighLander and .NET 2.0. Those who stay on .NET 1.x legacy
    platform cannot try GV, and HardQuery is the last version of CBC for
    them. Luckily, changes in GrapeVine is limited in LeXDK mainly, so the
    pluses updates may be able to ported back to .NET 1.1 and work well
    with LeXDK HardQuery (5.2).

    GrapeVine will bring in the following changes:
    \begin{itemize}
      \item FastInvoke method will be used to load pluses, which has been
      revealed to be much faster.

      \item Generics methods will be added to reduce duplicate code.

      \item Many .NET 2.0 BCL new classes will be used such as Stop\-watch
      and WebBrowser to replace a few custom classes.

      \item BDS OTA version 5 support will be added in order to load with
      BDS 2007/Highlander (BDS 5.0).

    \end{itemize}

    More pluses and minuses will be added. Current plan is to port in
    more features from SBT and add a Readme Plus. Since HighLander will
    be out next year, so these pluses and minuses may be completed even
    on HardQuery.

    Stay tuned.
  \end{quotation}

  \item 2006-08-05 Announce CBC HardQuery M4
  \begin{quotation}
    Last version I focused on refining LeXDK and made something really
    easier now. Yes, all pluses except NFamily is now okay. But that was
    still M3.

    Today I am glad that feature ComponentNamer Beta is done. Something
    wrong there seems to be an OTA bug (Now I am sure that it is an OTA
    bug. PropertyDescriptor class is not correctly implemented in .NET
    WinForms designer in BDS, so its SetValue method fails sometimes
    without any exception or information. It just fails to set the value
    to the object, as a result, ComponentNamer is not 100\% okay and
    fails sometimes). I traced in the code and found it was out of my
    scope. I do not know if I need to report this or David has reported
    it already. Other parts of WiseEditor Plus may be tested this time
    all in one. I wish when shipping, they works okay. This is the goal
    of M4.

    M5 will focus on NFamily. It is really a ''hard query'' because the
    code is not designed by me. Maybe some day I can ask David for help.

    Just a minute ago, I saw the resource not found exception when CBC was
    not loaded. OMG, it is a bug of the IDE possibly.
  \end{quotation}

  \item 2006-07-30 Announce CBC HardQuery M3
  \begin{quotation}
    Last version is not released but some important changes happen today,
    so I have to increase the version number a little bit.

    In the beginning, I was looking for some easier way to fill controls
    into Preferences Form. I failed because at that moment I did not know
    how to use User Control. It is something just like TFrame in VCL.
    Finally I know it today and make visually designing Tab Pages in Form
    Designer possible.

    Now, only Framework and CodeBeautifiers Plus are changed to meet this
    change of using User Control. I wish soon I will make all other pluses
    compatible. Thanks to this change, I adopt the SBT AStyle style preview
    function immediately. This way is more accurate than using ToolTips.
  \end{quotation}

  \item 2006-7-23 Announce CBC HardQuery M2 (not released)

  \begin{quotation}
    KyrSoft.CSBuilderGoodies.M\-in\-us is now merged into ThirdParty.
    Live Help window is now named AlertForm in Lextm Common Library. It
    is implemented by Mukund Pujari.

    In order to display HTML strings without any limitation, KyrSoft
    implemented QuickdocViewer is dropped. Instead, I use Pawel
    Glowacki's implementation. His implementation is better except that
    he makes use of Borland.mshtml assembly which is not redistributable.
    As a result, I add a getmshtml project, which automatically copies
    that assembly from BDS bin folder to CBC folders during installation.
    This workaround works fine.

    Also, I find what's wrong with HtmlControl. hc.control is null when
    that form is created and html segment is assigned, so no content is
    displayed. I don't know what causes this situation.

    Soon, I find that Microsoft.mshtml does just the same thing
    Borland.mshtml does. As a result, I use Microsoft.mshtml instead and
    now getmshtml is not used any more. It is sad to find that
    Microsoft.mshtml is also large. However, after some checking I see that
    it is already installed in GAC on two PCs in my lab. I guess it is
    included in .NET SDK, so I will check whether it is there on my test
    machine where only SDK is installed.

    At last, I use a MSHTML.dll I build with SDK tool, tlbimp. I think I
    have enough rights to redistribute it with CBC. Actually, anyone who
    gets .NET SDK installed could build this assembly by himself or
    herself. As a result, you know it replaces either Borland.mshtml.dll or
    Microsoft.mshtml.dll. Thanks to The Code Project, and who wrote that
    good article on it. I rather not have it shipped with CBC because many
    users who have Office 2003 installed may already have this assembly
    placed in GAC, since it is included in PIAs. However, other users may
    not have this PIA. So, I am forced to increase the size of CBC this
    time and you should forgive me.

    I will check SharpDevelop source to see if there is a good solution.

    SharpDevelop 1.1 was using a HtmlControl, but I do not want to see in
    to it. The main reason is about .NET 2.0 changes. In .NET 2.0, neither
    a HtmlControl nor a MSHTML.dll wrapper is needed, because there is a
    WebBrowser class and a HtmlDocument class. After checking those classes
    in 2.0 SDK documentation, I find it rather easy to make things done
    well in .NET 2.0. So when I have a workaround above in .NET 1.1, I do
    not have to do some smarter thing when a quite simple solution is there
    in .NET 2.0 waiting for me. So as to be compatible with HighLander, I
    will surely rewrite FormXmlDocViewer with WebBrowser and HtmlDocument
    later. This is my final decision.

    Also, I find it unnecessary to make a LeXDK for SharpDevelop. It is
    easy to extend SD. In about 30 minutes I wrote a AStyle AddIn (Beta).
    Although SD has included an indenter inside, it lacks of certain
    features like build-in styles implemented in AStyle. I guess this AddIn
    may be useful. I will publish its Beta version and see whether people
    like it.

    I use ASpell to check errors in these documents, and I find some
    conflicts. When I use --encoding=utf-8, find all errors, and quick fix
    them, Eclipse parses the file again and deadlocks. It is not hard to
    find that encoding is the cause. Today I change Eclipse's default
    encoding to utf-8 instead of GBK, then I find the bug is gone.

    These days I have been reading the book, ''Humane Interface''. Although
    my English skill is still poor and I cannot understand many terms and
    theories inside, I enjoy myself very much because the pictures in it
    already display many important design rules I must follow. Today I
    update CB Plus FormSaveFile. You can see the changes and feel that it
    is much easier to use this form.

    ToolTip is used to add Style Preview function to CB Plus AStyle tab
    page. It is a easy but painful way because in ToolTip class spaces in
    tip are compressed, and the style does not look exactly what it is. I
    think in order to gain best appearance, a previewer may be added later
    to display the styles.

    --indent-namespace option seems to be not necessary when generating
    AStyle parameter list. I will check AStyle manual again and make a hard
    decision.
  \end{quotation}

  \item 2006-7-16 Announce CBC HardQuery RC 1

  \begin{quotation}
    Now I remove the reference to csbgoodies.dll and include some files
    into a new minus named KyrSoft.CSBuilder\-Goodies.\-M\-in\-us. It
    provides a Live Help window to display HTML strings. I add some code
    in CodeBeautifiers Plus in order to test this Live Help feature. When
    Ctrl + W is pressed, a tip will show.

    Some UI items are modified. For example, the forms have accept
    buttons and cancel buttons this time. Also, License and Readme
    documents are updated.

    Now, I use Eclipse + TeXlipse instead of EditPlus to maintain these
    documents. It is fantastic and really platform independent. You may
    imagine that I am doing some modifications on Ubuntu Linux system.

    I will soon make InDate 2 complete and make Live Help more beautiful.

    I have found an article on CodeProject about FastInvoke method by
    reflection. However, this great method is implemented in .NET 2.0. CBC
    will switch from normal P/Invoke to FastInvoke in order to speed up
    loading.
  \end{quotation}

  \item 2006-7-1 Announce CBC NixNewNer Update 1

  \begin{quotation}
    \begin{enumerate}
      \item + InDate 2 is completed.
      \item * License.pdf is modified.
    \end{enumerate}
    Regular expression is now used to find matches in CBC's GForge
    homepage file list in InDate 2. The structure for this tool is also
    modified. A linked list is used to store those steps and make it
    easy to understand. What is left? I have not connect UpdateInfo
    unit to InDate's preferences.

    Currently, InDate 2 needs a lot of tests still. Once I find problems,
    hotfixes would be provided.

    For those Update 1 RCs users, I will update the old blog link and you
    can try whether InDate 1 works well.

    This update took about a month to finish because I was writing an
    article meanwhile. I wish this update is stable enough for users.

    What is more? License.pdf contains a lot of modifications so as to let
    you know more about this project's licensing issue. Since many other
    projects' source/binaries are used in CBC, the problem is complex and
    driving me crazy sometimes. I wish I have expressed myself correctly in
    that document.
  \end{quotation}

  \item 2006-6-24 Announce CBC NixNewNer Update 1 RC 6

  \begin{quotation}
    \begin{enumerate}
      \item * InDate is completed in RC 4.
      \item * Bug \#14 is fixed in RC 5.
      \item * Bug \#15 is fixed in this RC.
    \end{enumerate}
    There are no features added now. I have no time to do much work.
  \end{quotation}

  \item 2006-6-14 Announce CBC NixNewNer Update 1 RC 3

  \begin{quotation}
    \begin{enumerate}
      \item * .plus2 file format is done.
      \item + InDate feature is added.
    \end{enumerate}
    Although InDate feature is half-implemented, it can be used to
    update CBC now. .plus2 file format gives the user more power to
    configure features for every BDS versions. I wish I could make them
    work well before the final of Update 1 is released.

    RC 1 and RC 2 are fast fixes for N3 Final. They contain items listed
    above. But they are not mature.
  \end{quotation}
  \item 2006-6-8 Announce CBC NixNewNer

  \begin{quotation}
    \begin{enumerate}
      \item * Plus Manager GUI is updated.

      Use TreeView instead of ListView. Partially enabled Plus will be
      displayed in yellow color.

      \item * AutoCompletion is tuned again.
      \item + Adds some XML tag supports in C\# that you can invoke in
      Ctrl + J. (BDS 4 only)
      \item * \#13 bug on the bug tracker is fixed.
      \item - Input Helper is disabled.
      \item * AddMany Plus now works well under BDS 1.0.
      \item - FileWizards of WiseEditor Plus is disabled under BDS 1.0.
      \item * Documents are updated.
    \end{enumerate}

    Plus File Format Version II will be included in Update 1.

    Much effort is put on testing. Unstable features are disabled by
    default. Across BDS version compatibility is analysed this time. BDS
    1.0 and 2.0 supports are tested.

    Since I am stuck with BDS OTA bugs and vulnerabilities, I find it is
    not easy to deliver AutoCompletion and Input Helper before HardQuery
    (LeXDK version after NixNewNer).

    Stay tuned.
  \end{quotation}

  \item 2006-6-3 Announce CBC NixNewNer RC 1

  \begin{quotation}
    1. Plus Manager GUI is updated.

    2. AutoCompletion is tuned again.

    I decided not to release a WalkPace version because now LeXDK
    undertakes a lot of changes after I update the Plus Manager. Soon
    there will be more. So last version, WalkPace Update 2 RC 3 is the
    final.

    Finally I think the Utilities Plus is mature now. So I version it 1.0
    and start a new project codenamed WideLane. Plus Manager feature
    version 2 is the beginning. Why don't features have distinct icons as
    well as the pluses? It will be much more beautiful.

    Version number is now updated to 5.1. This is better indicating the
    changes in LeXDK is not quite compatible to 5.0. But I try my best to
    keep your migration smoother.
  \end{quotation}


  \item 2006-5-31 Announce CBC WalkPace Update 2 RC 3 (Released)

  \begin{quotation}
    This should have been the 10th version but actually it is similar to
    last, because I has just enough time to tune AutoCompletion and Input
    Helper. So I haven't changed the version number of WalkPace this time
    again.
  \end{quotation}

  \item 2006-5-28 Announce CBC WalkPace Update 2 RC 2 (Released)

  \begin{quotation}
    This was the ninth build of WalkPace.

    Finally I decided to break the planned SBT Plus into a few small
    pluses. The first plus ready was called WiseEditor Plus. The name was
    chosen because all features were ported from BeWise SBT. I redid the
    AutoCompletion code.

    Not all features in this new plus are available, because some tests
    are not done.

    Some custom changes are easy to find. For example, it is now easy to
    set a shortcut used SharedNames class. Expressions such as
    \code{private const int DefaultFileShortcut =\\ \tab SharedNames.CTRL
    + (int)Keys.W; //16471 } can be studied without difficulty.

    NAnt scripts are updated. TipOfTheDay and Lextm.AddM\-an\-y projects
    source are included. To build them, simply use the make.bat in their
    root folders.

    C\#Builder Goodies Doc Insight feature is disabled. I will soon
    combine this feature with CnPack IDE Wizard Input H\-elp\-er feature
    to do an enhanced feature.
  \end{quotation}
  \item 2006-5-22 Announce CBC WalkPace Update 2 RC 1 (Released)

  \begin{quotation}
    It was WP's eighth build, but I forgot to modify the assembly
    information in time and left ''WP V'' label there.
    \begin{itemize}
      \item Some class/function names changed.
      \item Tip of the Day feature added.
      \item LeXDK composition changed.
    \end{itemize}
  \end{quotation}
  \item 2006-5-1 Announce CBC WalkPace VII

  \begin{quotation}
    It is so easy now to port in SBT otas, because I keep so many
    things nearly the same. After a few minutes I ported in about five
    important parts of SBT, including NDoc, NANT, and so on.

    However, it is still hard to port in the configuration form. I need
    to find a good way.

    After more tests, I think it will be stable enough to be published.
    Because many tests are already done by David long time ago.

    Also inspired by the match man icon of China-Pub, I creates a new
    icon for LeXtudio.

    Nearly all SBT code are ported in. Tests are needed.
  \end{quotation}
  \item 2006-4-24 Announce CBC WalkPace Update 1 (5.0.1.1018)
  (Released)

  \begin{quotation}
    Licensing issues are changed. And also there are more features.

    Yes, some features are finished while others not.

    Here I update the information.

    Finished:

    1. a Plus Manager to manage Pluses and Features.

    2. an Expert Manager to manage both .NET and Win32 BDS plug-ins. It
    is an updated version of SBT's Expert Manager.

    ** The above file "lextm.utilities.plus.exe" can also execute as a
    stand alone executable. By default, it is the Expert Manager. If -p
    parameter is used, it will be the Plus Manager. Isn't it wonder?

    3. New Help menus added to BDS's Help menu.

    4. Updated manuals

    5. New Code Beautifiers Plus codenamed Long Jeans.

    6. New installer.

    7. Mix for BDS, which is a group of useful BDS shortcuts.

    8. More libraries added like ThirdParty.dll.

    Not Ready:

    1. Use ShineOn library to port in some Delphi for Win32 code.

    2. Comments for imported SBT code.

    See the Plus versions. 0.9.0.0 indicates that a Beta version it is.
    Actually only CB Plus is stable enough.

    And at last I started to use GForge to host this project. Visit it
    frequently and I will show new betas there.
  \end{quotation}
  \item 2006-3-12 Announce CBC WalkPace II (5.0.0.1018) (Released)

  \begin{quotation}
    AddMany 4.1 functions are now ported. It is also the first Delphi
    for .NET project I ported. Although there were a few annoying bugs
    popping up, I succeeded. It serves as a good demo of mixing C\# and
    Delphi for .NET code. Also it is a good demo to show that LeXtudio
    OpenTools SDK (LeXDK, for short) is easy to use.

    Sharp\-Builder\-Tools.Com\-mon files are moved to
    Lextm.Code\-Beauti\-fier\-Col\-lection.Com\-mon.

    I modified them using information in this email. It is from David
    Hervieux.

    \begin{quote}
      Hi here is a fix for SBT, I plan to do a release but I don't know
      when

      ----- Original Message -----

      From: Tomek Pi��rkowski \\
      To: dhervieux@users.sourceforge.net \\
      Sent: Sunday, February 20, 2005 6:04 AM \\
      Subject: [dev] SbTools and Delphi 2k5


      Hi,



      In Delphi 2005, SbTools don't work - you don't adapted yet sbt for
      this new Borland IDE.

      I correct sbt 3.1 sources to run in Delhi 2k5.



      Main problem, getting Borland IDE registry key, adequate to BDS
      version.

      It is possible to get this key form IOTAService, so this be
      independent of BDS version.


      \code{ Private string ideRegKey = "";\\
      public static string IDERegKey\{\\

      \tab      get \\

      \tab      \{\\

      \tab \tab           if( ideRegKey == "" )\\

      \tab \tab           \{\\

      \tab \tab                  ideRegKey =
      GetService().BaseRegistryKey;\\

      \tab \tab                   if( ideRegKey.Substring(
      ideRegKey.Length - 1, 1 ) != "$\backslash$$\backslash$" )\\

      \tab \tab                   \{\\

      \tab \tab                         ideRegKey +=
      "$\backslash$$\backslash$";\\

      \tab \tab                   \}\\

      \tab \tab             \}\\

      \tab \tab             return ideRegKey;\\

      \tab       \}\\

      \}\\
      }


      I attach this corrections with this message.

      After this alterations, sbt work with delphi 2k5.



      I attach also my InheritanceWizard, to generate new cs modules for
      subclassing. Maybe you use it in sbTools. Is a small problem with
      this wizard, I don't know, it is possible to create 2 files to new
      module (cs and aspx) to webcontrols new modules. It's not problem
      for me, because, I don't use WebForms, but generally problem
      remain. Files *Template.cs to put in the
      sbTolsBinDir$\backslash$Template$\backslash$Inheritance folder.



      Thanks for sbTools



      tom
    \end{quote}

    It is Tom's good idea that SBT can be run on Delphi 2005 and BDS
    2006. However, he failed to notice that there is an OTA bug in
    C\#Bu\-ild\-er 1.0 that prevents such modified version of SBT from
    running. (GetService().Base\-Reg\-istry\-Key returns a dirty string
    starting with an annoying ''$\backslash$'' ahead.) When I used his
    idea in CBC, I modified this property further to work around known
    issues (refer to OTAUtils unit to see details).

    Reflection is now used by the framework to load features (such as
    OtaAddMany and OtaCSBuilderGoodies). I also invent a special file
    format to store feature information in CBC 2 plugin assemblies.
    However, it is not completed yet. Next week I will solve it.

    I think it is good time to close the feature set of this Beta
    although I list a few more features in TODO. Before releasing CBC 2
    WalkPace Final more documents should be written and more tests
    should be carried out. They need time.

    In the future, two different distributions will ship for CBC 2. One
    is the Basic distribution, which contains only the framework
    feature, the code beautifiers feature and the LeXDK. The other is
    called Deluxe distribution that not only includes basic features
    but more features as well. The Basic version is for those who only
    needs a formatter. The Deluxe version is for those who ask for more.

    Also this is the first time I begin to release update for older
    versions. For example, last version CBC WalkPace I is made public
    as CBC BigFace Update 1. This is not my intention but I guess BF
    users may like to see it. I don't know whether it is a good way.
    This update can only be installed with BF already installed because
    a few supporting files are not included in the installer to reduce
    its size.

    Since I finally removed dependency on BDS version, I think it is
    time to bring CBC to other BDS version like Delphi 2005. I will
    do it lately before shipping this build.

    Added on 2006-3-19:

    It is neither too easy nor too hard to break this whole project
    further to a bunch of six assemblies. However, this enables me to
    layer this software as well as the LeXDK. This separation surely
    reduces coupling between classes.

    However, I only apply a few Design Patterns from the GoF Bible. I
    believe further learning and practice can make a more stable and
    flexible LeXDK.

    Also I make two solution for Basic and Plus Pack.

    Some files will not be shipped with this WalkPace Final version.
    The version number is now changed to 5.0.0.1018 so as to reflect
    its distinctions from the old 4.5 series. Next version should be a
    NixNewNer.

    The currently unavailable files are Plus Pack version 2006-3-19,
    LeXDK DG, and a few articles. They will be available on the
    homepage of CBC as soon as I finish.
  \end{quotation}

  \item 2006-3-5 Announce CBC WalkPace I (4.5.1.*)

  \begin{quotation}
    Architectural changes are finished. Shortcut settings are back
    with brand new design. Menu items with a non-zero default
    shortcut value now can be customized in FormShortcuts (so, more
    items than ever).

    XML indentation can not be changed in CBC FormOptions from this
    version on because now XMLLex reads BDS's settings for XML files.
    If you want to indent XML as you wish, you should changes BDS's
    settings, either by RegEdit or BDS itself.

  \end{quotation}

  \item 2006-2-12 Announce CBC BigFace II (2.4 RC 4 or Beta 4)
  (Released) (Added on 2006-3-5)

  \begin{quotation}
    Since when testing BigFace I found a few bugs that need urgent
    attention, I stopped to do some patches. However, at that moment I
    already changed the version number to 4.4, so I decided not to
    change it back. If you were careful, you may see some WalkPace
    features as well as Code Name WP in this release. However, it is
    still BigFace version in my eyes.

    I had ported many BeWise files to conform to my files. After porting
    GPL headers were added.

    Shortcut and XML indentation cannot be set in this release. They will
    be back very soon in WP 1 (you can change it in code and build
    manually).
  \end{quotation}
  \item 2006-2-7 Announce CBC BigFace I (2.4 RC 3 or Beta 3)

  \begin{quotation}
    This is not the BigFace mentioned in the Roadmap because finally I
    decided to implement the style designer (maybe named JCFHelper) in
    WalkPace. However, this version has a few annoying bugs fixed (for
    example, the updated XML formatter, XMLLex, will not eat your XML
    files when they are incomplete). The Roadmap will be updated soon.

    I did a lot to learn Design Patterns and finally found some way to
    update SBT 3.1 architecture a bit. You can find a lot patterns and
    data structures I use in this version so as to reimplement certain
    important functions (mentioned last time, OtaCodeBeautifier unit
    has a lot of changes. Actually Factory Method and Singleton
    Patterns are used now.) When this kind of refactorings were done,
    BigFace would go out and reach you.

    After these refactoring I believe this architecture has been improved
    much and allows me to bring in more features later.

    A lot of thanks go to NUnit. Now I can use this excellent tool after
    some trials. It is really a powerful tool. Believe me, without NUnit,
    I would not have done these important refactorings this time.
  \end{quotation}

  \item 2005-12-25 Announce CBC 2.4 RC 2 (Released as Beta 2)

  \begin{quotation}
    I have posted a roadmap on my personal blog. Yes, it is a new
    version done today with a few user-friendly improvements like a
    setter after installation and bundled AStyle and JCF executables.
    However, these improvements are not listed in the roadmap.

    After some painful refactoring I improved the Kiviat Chart for
    OtaCodeBeautifier class a lot. So I have to confess that Together
    for C\# has been a great tool in BDS 4. The speed of CBC must have
    been quicker. A few bottlenecks have been removed. Even though now
    only a few Singleton Design Patterns occurs in CBC, I find that it
    is much easier to understand the code now. Go lextm.
  \end{quotation}

  \item 2005-12-10 Announce CBC 2.4 RC 1 (Released as Beta 1)

  \begin{quotation}
    I hadn't include an announcement of 2.3 because I was not happy
    that day. Yes, I messed things up then. Version 2.3 was just an
    urgent fix for 2.2.

    I have to confess that since this December, I feel that I cannot
    contribute as much energy as usual to this little project. So,
    after this release I may take a good rest. Also, I should do more
    preparations of my coming career and find a good job next year.

    CBC 2.4 is a version that solely supports Delphi 2006 (BDS 4.0).
    Delphi 2005 users should use 2.3. Actually the cores are the same.
    Enjoy coding!

    Added on 2005-12-16: Inno Setup 5 and ISTool help me a lot when I
    make an installer for CBC. And I guess I can say farewell to the
    dear InstallShield now and forever. So all CBC users can enjoy an
    much easier way to install/uninstall CBC. Sorry that I know of Inno
    Setup so late :-)
  \end{quotation}

  \item 2005-11-18 Announce CBC 2.2 RC 3 (Released)

  \begin{quotation}
    Since I decided to use Microsoft FxCop (Oh it is OPEN SOURCE) to
    analyse my code (except original SBT part). So you may see a lot of
    change now in this version. Yes, I learned a lot during this
    period. At first I was following my own style formed in these
    years, but now I choose a more popular (maybe not that popular)
    way. It may be called some kind of standardization.

    Also I knew that I could not change a WinForm's initialComponents
    function now if I use an IDE. So when I do some refactoring with that
    function, I should take care. Yes, a hint about CBC3. Yeah, the
    architecture will be modified a bit then, I think. It is a good
    architecture that David implemented but years goes by. Maybe I can
    make some modifications in order to support Delphi 2006. It is now
    that I think I have the ability for the first time. So I should start
    soon. But you know it is end of this year these days. And I find it
    really hard to concentrate on my projects now.

    The icons are modified with the help of GIMP. They should be of
    better quality now.

    A beta version of installer is added. It only works for BDS 3 now,
    and should be able to install/uninstall the release version of CBC 2.

    If you want a better CBC, I beg you e-mail to me so as to remind me
    that I should work harder than now. Thanks.
  \end{quotation}

  \item 2005-11-13 Announce CBC 2.2 RC 2

  \begin{quotation}
    Update a few places and add a project for Delphi 8 for .NET. Now I
    officially support that IDE. Since the OTA version of it is still
    7.1 (nearly the same as C\#Builder 1.0). Yes it is enhanced but not
    as mature as Delphi 2005's OTA.

    I make a decision to do some important refactoring in the next
    version, so this should be a last version of 2 series. And thus
    next version should be 3.0. However, I still work on this version
    if it is required by the users.

    Also I found that Marc Clifton's CodeProject article "A Treatise on
    ..." was really bright and carefully designed. Now I use his code in
    my project to correct the bug numbered 11.

  \end{quotation}

  \item 2005-10-30 Announce CBC 2.2 RC 1

  \begin{quotation}
    Forgot to write down something about CBC 2.1's release on last
    Friday. It should be something sad. Since I have done a lot before
    the day except using it myself for a few day to find more problems.

    The Debug Builds should work well, but they give out log files. And
    the Release Build cannot be used at all. That is my business. I
    used the Debug Build myself so I forgot to change the settings to
    Release Build. I think next time I will do better.

    Thus, actually the version I released that day was really the 2.1 RC
    4. This time may be I should release RC 5 or RC 6. Who knows?

    Today I add an installer to replace SBT's Expert Manager. It is
    simple. And yesterday, I change the option dialog a bit. Now I should
    add some icons, too. Also I correct the Build settings. That is
    important. Maybe I should fix this on BBS.cn\-pack.org and BDN soon.

    Now icons are loaded from the file directly. I give up the way ArtCSB
    uses. To use a resource file is much complex.

  \end{quotation}

  \item 2005-10-26 Announce CBC 2.1 RC 4 (Released)

  \begin{quotation}
    I finally make it easy to change the shortcut these days. And a few
    tests are in advance on Delphi 2005. So a more stable version it is
    now.

    What bugs remaining is still the main menu. I don't know how
    Borland R\&D've done their plug-in like StarTeam and Together. But
    it is not a big matter, so I decide to solve it later.

    I am also waiting for the reply of ArtCSB's author. I need his
    permission to avoid future legal issues. :-)

    I believe the 2.1 version will be out soon.

    And Yahoo told me that I cannot reach the author of ArtCSB now. The
    e-mail was returned.

    I think I may release 2.1 without his permission, if I declare my
    acknowledgement.

  \end{quotation}

  \item 2005-10-22 Announce CBC 2.1 RC 3

  \begin{quotation}
    I don't know when 2.1 will be out now. It seemed that I tried to
    make it more stable every day. So I feel that maybe tomorrow a new
    bug will occur. Maybe Code Review is necessary now.

    First, the XML formatter was back, but now I add messages to it,
    and test it outside the plug-in. I choose some NUnit way, but I
    don't use NUnit, but a console written by me. The test is easiest,
    because I use CodeDom from MS as the technique core, so further
    tests are unnecessary. What I focus is whether it works.

    Second, a few separators are added, so when the plug-in menus
    displayed under Tools or CnPack menu, a clean UI is there.

    Third, I realise the idea that I make a few available build-in styles
    for JCF. Now, basic structure is set up. What is needed are settings
    files. So if you want a special style or you have something to
    contribute, send e-mail to me.

    Fourth, according to ArtCSB 2.5's source, I add all things about
    IDEUnregister. And since I use materials from BDN's article on
    C\#Builder 1.0's .NET OTA, now the shortcut refreshing problem is
    solved. What is remaining is that the menu is still different from
    others. I should have checked the solution on CnPack's BBS.

  \end{quotation}
  \item 2005-10-20 Announce CBC 2.1 RC 2

  \begin{quotation}
    Minor changes occur to result a more stable version before it is
    released.

    First, XML formatter is back (though still buggy).

    Second, change a few OTA calls to be compatible with ToolsAPI 7.1
    (BDS 1.0).

    Third, one more position available for you to place CBC menus. That
    is under CnPack menu. I add it because I still care for those
    CnPack users who demand a formatter feature in CnPack. Let this
    wizard be a alternative way. Also, here I add a new validate line
    to verify that the position is valid before inserting CBC menus.

    Fourth, add a data folder to hold data used by CBC, and place files
    there in a new order, thus OptionManager.cs has been heavily changed.
    Rename AssemblyInfo.cs to PlugInInfo.cs to reduced confusion.

  \end{quotation}
  \item 2005-10-16 Announce CBC 2.1 RC 1

  \begin{quotation}
    CBC should be called "the ultimate code beautifiers" now, I
    believe. And Borland finally decides to release DeXter in early
    December. A good news for BCB users.

    After applying a lot of refactoring principles, now CBC has evolved
    to be SBT 3.1.0.0 compatible. That architecture is quite beautiful
    if you dive in (although there is still work ahead). And David has
    already replied to my e-mail, and permitted me to use his code.
    "Everything's gonna be all right".

    Other major changes are listed in the readme file.

    I have done a lot of enhancements for the documents and source
    structure. It is the first time that I add a short description about
    me, the author of this plug-in.

    Now there is a new folder named "technotes" in the "docs" directory
    for those ideas flashed into my mind suddenly. I wish they would be
    useful for the beginners to understand the process I develop this
    piece of software. For experts, a lot of files about the techniques
    I've used are included also. For instance, there are files about SBT
    architecture, and BDS OTA for .NET.

    Big thanks to CBC itself. I've found a good chance to make use of
    those software engineering books, articles, and tools here.

    Also I pulled out some memory on JCF Expert 1.0 and put them at the
    beginning of this file (I should've started to write about that long
    before).

    There are still things to do, such as internationalization of CBC
    (currently only English version is available). I'm learning about
    this technique these days.

  \end{quotation}
  \item 2005-9-15 Announce CBC 2.0 (Released)

  \begin{quotation}
    Delphi 2005 is named the 'ultimate Delphi', but I guess (and proved
    to be true that) the next version of BDS should be another
    milestone. In the version 2006, C++Builder would be included. So,
    totally 4 languages would be in BDS IDE.

    Luckily JCF can format both of the Delphi languages (Win32 and
    .NET), and AStyle is capable of C++/C\#. As a result, I decide to
    make an expert based on SBT's code (where a good plug-in framework
    is built in) which integrates these two command-line tools into
    Delphi 200X's IDE.

    The reasons are simple:

    First, I use so many languages and need beautifiers' help;

    Second, SBT is designed mainly for C\# developers, so only C\# source
    code are considered.

    C\# will be used to program this expert, because I referred to part
    of SBT v0.9 (David, I owe you so much), and also I like this
    language, who is another son of Anders.

    Code Beautifier Collection Expert, action now!

  \end{quotation}
  \item 2005-3-19 Announce JCF Integration Expert v 1.0.1 (Released)
  (Added on 2005-10-16)

  \begin{quotation}
    The software was something derived from SBT v0.9. I didn't choose
    SBT v3.1.0.0 because I didn't understand that new architecture very
    well then. After learning more about Design Patterns, I use SBT
    v3.1.0.0's architecture later.

    The long name was a mimic of ModelMakerTools' plug-in for
    ModelMaker 8.

    This piece of software was written for personal use in the beginning,
    but soon released to public at 2ccc.com.

    As twin of this software, I wrote a patch to SBT v3.1.0.0 in order to
    add JCF support. That was the basis of later CBC. However, at that
    moment, David, author of SBT decided to stop updating SBT, so it is
    impossible for the users to see my patch packed in SBT's new version.

    Later (about two months later, in May), I found a Delphi native open
    source plug-in named CnPack. I joined that team, but my
    Delphi-version of JCF and AStyle patch (available on
    BBS.cn\-pack.org) was not finally accepted by that team. It is a pity
    since I made a lot of efforts to move my C\# code to Delphi code, and
    moved from SBT architecture to CnPack's.

    This made me feel that I should update JCF Expert in some ways.

    The trigger of CBC was a letter from my first known user. He
    downloaded the plug-in from 2ccc.com, and e-mail to me. He asked for
    a new version, and I just had update the piece I was using those
    days. So, everything was OK for CBC to be born.

  \end{quotation}

\end{itemize}
