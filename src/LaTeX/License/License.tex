\documentclass{article}
\usepackage{makeidx}
\makeindex

\usepackage{graphicx}           %ͼƬ

\usepackage{CJK}                %����
\usepackage{CJK,CJKnumb, CJKpunct}
%% CJKulem}         % ����֧�ֺ��
\usepackage{indentfirst}        %��������
\usepackage{float}              %������
\usepackage[usenames]{color}    %��ɫ
\usepackage{amssymb}            %��ѧ����

%%\newlength\CJKtwospaces
%%\def\CJKindent{%
%%\settowidth\CJKtwospaces{\CJKchar{"0A1}{"0A1}\CJKchar{"0A1}{"0A1}}%
%%\parindent\CJKtwospaces}

\setlength{\parindent}{20pt}
\newcommand{\pic}[3]{
	\begin{figure}[!htp]
		\begin{center}
			\includegraphics[#3]{images/#1}
			\caption{#2}
		\end{center}
	\end{figure}
}
\newcommand{\picnocaption}[1]{
	\begin{figure}[!htp]
		\begin{center}
			\includegraphics{images/#1}
		\end{center}
	\end{figure}
}
%% cbc picture
\newcommand{\piccbc} {
	\picnocaption{cbc36}{}
}

%% lextm
\newcommand{\lextm} {\mbox{Lex Y. Li~}}

%% emph
\newcommand{\key}[1] {
	\mbox{\emph{#1}}
}
%% boxed paragraph
\newcommand{\boxing}[1] {
	%% \framebox[\textwidth][r]{#1}\par
	%% \parbox[c]{8cm}{#1}
	%% \begin{minipage}[r]{8cm} #1 \end{minipage}
	\begin{small}
	\begin{verse}
	{\em #1}
	\end{verse}
	\end{small}
}
%% For source code
\newcommand{\code}[1] {
	\begin{small}
	\begin{quote}
		\begin{flushleft}
			\begin{sffamily}
			#1
			\end{sffamily}
		\end{flushleft}
	\end{quote}
	\end{small}
}
%% Tab key alternative
\newcommand{\tab} {\ \ \ \ }

%% Hyperlink
\newcommand{\hyperlink}[1] {
	\begin{center}
	\underline{#1}
	\end{center}
}

\newcommand{\copyclaimer} {
\newpage


\begin{flushleft}
Content is available under GNU Free Documentation License 1.2.


Copyright \copyright\ 2005-2007, \lextm. All Rights Reserved.
\end{flushleft}
\begin{center}
Powered by \LaTeX.
\end{center}

\begin{center}
Disclaimer
\end{center}
\lextm has taken due care in preparing this manual and the programs and data on the electronic disk media (if any) accompanying this
book including research, development and testing to ascertain their effectiveness.

\lextm makes no warranties as to the contents of this manual and specifically disclaims any implied warranties of merchantability or fitness
for any particular purpose. \lextm further reserves the right to make changes to the specifications of the program and contents of
the manual without obligation to notify any person or organization of such changes.

Newest version of this manual should be shipped in the distribution package but is not guaranteed.



\begin{center}
Legal Notice
\end{center}
lextm, 
LeXtudio, 
LeXDK,
Code Beautifier Collection, 
and other LeXtudio marks are owned by \lextm and may be registered. 
All other trademarks are the property of their respective owners.
	
}

\renewcommand{\baselinestretch}{1.5}

\author{\lextm}
\title{Code Beautifier Collection License Issues}

\begin{document}

\maketitle

\copyclaimer

\newpage

\begin{flushleft}
This file contains important supplementary information that may not appear in the main product documentation. \lextm recommends that you read this file in its entirety.

For information about licensing issues, see other files located, by default, at C:$\backslash$Program Files$\backslash$Code Beautifier Collection$\backslash$doc.

\end{flushleft}

\newpage

\pagenumbering{Roman}
\setcounter{page}{1}

\tableofcontents

\newpage

\pagenumbering{arabic}
\setcounter{page}{1}

\section{General Notes}
Code Beautifier Collection Standard is doned by \lextm\ and other
contributors. Portions created by a specific author are copyrighted by him/her.

It consists of several assemblies. Each assembly/plug-in is licensed under its
respective license terms.

If there is something wrong or you have a problem, please contact me by sending
a mail to \underline{lextudio@gmail.com}.

\section{Assemblies Licensing}

\subsection{The Framework Assembly}

All files in the Lextm.Code\-Beautifier\-Collection.Frame\-work assembly is
licensed under the terms of GNU General Public License.

The Framework assembly uses the third party library 
UnhandledExceptionManager, which is included as binaries (DLLs).

\subsubsection{UnhandledExceptionManager V2.0 by Mauro Venturini}
When a .NET application hits an unhandled exception the default behaviour is to
show a very poor dialog (Form1.jpg). Therefore, it is (or should be) common
practice to hook UnhandledException event of the current AppDomain and
ThreadException event of Application. Unfortunately, there are security
restrictions on these hookings that prevent their direct use inside
applications that will be run without full trust (and even Local Intranet group
does NOT have full trust). A solution is to delegate the hookings to a full
trusted assembly inside the GAC and reference it. UnhandledExceptionManager is
an assembly that implements the technique.

UnhandledExceptionManager.dll was created and tested using Borland Developer
Studio V10.0.2166.28377 (Delhi 2006 Update 1) under Windows 2000 SP4,.NET 1.1
SP1 Merging of Borland.Delphi.dll was done using DILMerge. DILMerge can be
dowloaded at http://cc.borland.com/Item.aspx?id=23227

\underline{Please notice the version included in CBC is a slightly modified version.}

\subsection{LeXDK Assembly}
Most of Le\-xtm.Le\-XDK.Co\-re files are licensed under the terms of GNU Lesser General Public
License.

A few files are taken from SBT directly, so they are copyrighted by David
Hervieux. Refer to the file headers and appendix \ref{sec:sbt} for
details.

\subsection{BeWise.SharpBuilderTools.Minus}
This assembly is built on SBT code with a few modifications done by \lextm.
Original SBT files are copyrighted by David Hervieux. Refer to the file headers and
appendix \ref{sec:sbt} for details.

\subsection{Utilities Plus, CodeBeautifiers Plus, BagPlug Plus, CSBuilderGoodies
Plus, and AddMany Plus}
They are written by \lextm\ and licensed under the terms of GNU General Public
License.

CSBuilderGoodies Plus brought the idea of C\#Builder Goodies 1.1 assembly. One
original file ''QuickdocViewer.cs'' is ported and modified by \lextm.
Original source code is copyrighted by Valentino Kyriakides (KyrSoft).

AddMany Plus uses the third party library AddMany, which are included as
binaries (DLLs). Refer to appendix \ref{sec:addmany} for details on AddMany.

Utilities Plus uses the following third party libraries which are included as
binaries (DLLs):

\begin{itemize}

  \item SharpZipLib library.

  \item ThirdParty assembly. Refer to appendix \ref{sec:thirdparty} for details.

%   \item Invisibles library.

  \item LoadingCircle control.

\end{itemize}

\subsubsection{C\#Builder Goodies (written by Valentino Kyriakides)}
The C\#Builder Goodies OpenTool is an add-on module for the Borland C\#Builder
IDE which helps you to perform C\# language specific reserved keyword and XML
documentation autocompletions inside the C\#Builder editors. Further it
actually also includes a quick access WYSIWYG XML to HTML viewer helper and a
keycode analyzer helper. The main motivation for this C\#Builder OpenTool is,
to enhance the productivity of lazy typers by offering some of missing
C\#Builder support for C\# reserved keywords and XML documentation tags
autocompletion.

Copyright \copyright\ 2003, Valentino Kyriakides (KyrSoft)

Available on the web at
\hyperlink{http://cc.borland.com/Item.aspx?ID=21168}


% \subsubsection{Lextm.AddMany}
%
% It contains only a wrapping class that connects AddMany 4.1 to CBC. This class
% is a slightly modified version of AddMany TIDEPlugin unit. Since Mauro
% Venturini does not mention any restriction, I do not add any restriction to
% this modified version, too.
%
% The code is public domain, you may use and modify it freely.
%
% Refer to appendix \ref{sec:addmany} for details on AddMany.

\subsubsection{ICSharpCode.SharpZipLib Library}

SharpZipLib is used in Utilities Plus' InDate feature to unzip zip package.

This library is under GPL. You can see the license details on its homepage.

Available on the web at
\hyperlink{http://www.icsharpcode.net/OpenSource/SharpZipLib/Default.aspx}

% \subsubsection{Invisibles library by Mauro Venturini}
% Windows Forms applications often require some sort of asynchronous invocation
% option. .NET 2.0 provides the BackgroundWorker component to facilitate easy
% asynchronous invocation with Windows Forms. Waiting for it this is a Delphi.NET
% BackgroundWorker component implementation based on .NET 1.1.

\subsubsection{LoadingCircle control by Martin R. Gagn$\acute{e}$}
\begin{small}
\begin{flushleft}
//\\
// Copyright \copyright 2006, Martin R. Gagn$\acute{e}$ (martingagne@gmail.com)\\
// All rights reserved.\\
//\\
// Redistribution and use in source and binary forms, with or without modification, \\
// are permitted provided that the following conditions are met:\\
//\\
//   - Redistributions of source code must retain the above copyright notice, \\
//     this list of conditions and the following disclaimer.\\
//\\
//   - Redistributions in binary form must reproduce the above copyright notice, \\
//     this list of conditions and the following disclaimer in the documentation \\
//     and/or other materials provided with the distribution.\\
// \\
// THIS SOFTWARE IS PROVIDED BY THE COPYRIGHT HOLDERS AND CONTRIBUTORS "AS IS" AND \\
// ANY EXPRESS OR IMPLIED WARRANTIES, INCLUDING, BUT NOT LIMITED TO, THE IMPLIED \\
// WARRANTIES OF MERCHANTABILITY AND FITNESS FOR A PARTICULAR PURPOSE ARE DISCLAIMED. \\
// IN NO EVENT SHALL THE COPYRIGHT OWNER OR CONTRIBUTORS BE LIABLE FOR ANY DIRECT, \\
// INDIRECT, INCIDENTAL, SPECIAL, EXEMPLARY, OR CONSEQUENTIAL DAMAGES (INCLUDING, BUT \\
// NOT LIMITED TO, PROCUREMENT OF SUBSTITUTE GOODS OR SERVICES; LOSS OF USE, DATA, \\
// OR PROFITS; OR BUSINESS INTERRUPTION) HOWEVER CAUSED AND ON ANY THEORY OF LIABILITY, \\
// WHETHER IN CONTRACT, STRICT LIABILITY, OR TORT (INCLUDING NEGLIGENCE OR OTHERWISE) \\
// ARISING IN ANY WAY OUT OF THE USE OF THIS SOFTWARE, EVEN IF ADVISED OF THE POSSIBILITY \\
// OF SUCH DAMAGE.\\
//\\
\end{flushleft}
\end{small}

\subsection{NFamily Plus and WiseEditor Plus}
They are features taken directly from SBT or modified by \lextm. SBT files are
copyrighted by David Hervieux. New files are copyrighted by \lextm.

% \boxing{It was my intention that features I created should be covered by GPL,
% but I had no time to extract them from WiseEditor Plus. I would move them to a
% separate plus later, for example, Source Navigator. So remember if you want to
% use these specific files, see the headers please. I added GPL headers to them
% already.}

WiseEditor Plus uses the following third party libraries which are included as
binaries (DLLs):

\begin{itemize}
  \item ThirdParty assembly. Refer to appendix \ref{sec:thirdparty} for details.

  \item NetSpell library.
\end{itemize}

NFamily Plus uses the following third party libraries which are included as
binaries (DLLs):

\begin{itemize}
  \item ThirdParty assembly. Refer to appendix \ref{sec:thirdparty} for details.
  \item SMS.Windows.Forms assembly.
  \item NUnit Framework assembly.
\end{itemize}

\subsubsection{NetSpell library}
The NetSpell project is a spell checking engine written entirely in managed C\#
.net code.  NetSpell's suggestions for a misspelled word are generated using
phonetic (sounds like) matching and ranked by a typographical score (looks
like).  NetSpell supports multiple languages and the dictionaries are based on
the OpenOffice Affix compression format. The library can be used in Windows or
Web Form projects. The download includes an English dictionary with
dictionaries for other languages available for download on the project web
site. NetSpell also supports user added words and automatic creation of user
dictionaries. The package includes a dictionary build tool to build custom
dictionaries.


BSD License
\begin{quote}
Copyright (c) 2003, Paul Welter\\
All rights reserved.


  Redistribution and use in source and binary forms, with or without
  modification, are permitted provided that the following conditions are met:

  1) Redistributions of source code must retain the above copyright notice,
  this list of conditions and the following disclaimer.

  2) Redistributions in binary form must reproduce the above copyright notice,
  this list of conditions and the following disclaimer in the documentation
  and/or other materials provided with the distribution.

  3) Neither the name of the ORGANIZATION nor the names of its contributors may
  be used to endorse or promote products derived from this software without
  specific prior written permission.

  THIS SOFTWARE IS PROVIDED BY THE COPYRIGHT HOLDERS AND CONTRIBUTORS "AS IS"
  AND ANY EXPRESS OR IMPLIED WARRANTIES, INCLUDING, BUT NOT LIMITED TO, THE
  IMPLIED WARRANTIES OF MERCHANTABILITY AND FITNESS FOR A PARTICULAR PURPOSE
  ARE DISCLAIMED. IN NO EVENT SHALL THE COPYRIGHT OWNER OR CONTRIBUTORS BE
  LIABLE FOR ANY DIRECT, INDIRECT, INCIDENTAL, SPECIAL, EXEMPLARY, OR
  CONSEQUENTI\-AL DAMAGES (INCLUDING, BUT NOT LIMITED TO, PROCUREMENT OF
  SUBSTITUTE GOODS OR SERVICES; LOSS OF USE, DATA, OR PROFITS; OR BUSINESS
  INTERRUPTION) HOWEVER CAUSED AND ON ANY THEORY OF LIABILITY, WHE\-THER IN
  CONTRACT, STRICT LIABILITY, OR TORT (INCLUDING NEGLIGENCE OR OTHERWISE)
  ARISING IN ANY WAY OUT OF THE USE OF THIS SOFTWARE, EVEN IF ADVISED OF THE
  POSSIBILITY OF SUCH DAMAGE.

\end{quote}
Available on the web at
\hyperlink{http://www.loresoft.com/Applications/NetSpell/default.aspx}

\subsubsection{SMS.Windows.Forms library}
This library is created by Steven Soloff. Available on the web at

\hyperlink{www.codeguru.com/c\-sh\-arp/csh\-arp/\-cs\_con\-trols/\-wi\-zards/\-ar\-ticle.php/\-c4799/ }

\subsubsection{NUnit Framework}
NUnit Framework is under zlib/libpng License.

Available on the web at
\hyperlink{http://sourceforge.net/projects/nunit/}



\subsection{ArtCSB Plus}
It is written by \lextm. ArtCSB Plus consumes a few files of ArtCSB 2.5 which
are copyrighted by Artyom Fedyuk. Other files are modified and copyrighted by
\lextm.

\subsubsection{ArtCSB (written by Artyom Fedyuk)}
ArtCSB 2.5 is an addition to Borland C\# Builder 1.0 environment. Also you can
install ArtCSB 2.5 to Borland Delphi 8 IDE.

This product realize next new menu items in IDE:

1. "View/ArtCSB Source Navigator" is intended for fast navigation in C\# -file.
It includes an opportunity of sorting program objects in different ways. It can
display an objects tree in the extraction mode. The extraction mode displays
the source line that contains the object definition instead of a simple object
name. Source Navigator can also save object structure to a html or text file.

2. "View/Designer Navigator" is intended for tracking the tree of components
located on the forms (Windows and ASP.NET applications). The purpose of
creating Designer Navigator was to realize an expert similar to Delphi 7 Object
TreeView. Like in Delphi 7 Object TreeView, you can select objects on the form,
thus selecting corresponding items in Designer Navigator. When you select items
in Designer Navigator, corresponding objects on the form will be selected too.
Multiple selections are possible (by pressing key Ctrl or Shift).

3. "View/ArtCSB Editor Outlining" is intended for handy elide/unelide
functions. Draw an attention on powerful "Elide to definitions"-item.

4. "View/ArtCSB View bdsproj as xml" is intended for viewing main
bdsproj-project file as xml-file. It creates xml-copy of main bdsproj-project
file in project directory and opens it.

5. "Tools/ArtCSB Auto-Save Options" is intended for periodically project
autosaving in Borland IDE.

Available on the web at
\hyperlink{http://cc.borland.com/Item.aspx?id=21908}


\section{References}
Tips from several articles are used in CBC development.

\begin{itemize}
%   \item C\#Builder XML Documentation Viewer Wizard, Pawel Glowacki, BorCon 2003
%   Poster. I follow Pawel's way to build the viewer UI.

  % \item A Treatise on Using Debug and Trace classes, including Exception
  % Handling, Marc Clifton, The Code Project. I modified the unit provided and use
  % it heavily.

%   \item Some Cool Tips for .NET, Mukund Pujari, The Code Project. I use the
%   alert form.

  \item How to write a loading circle animation in .NET?, Martin Gagn$\acute{e}$, The
  Code Project. Updated: 15 Feb 2007

  \item A deep XmlSerializer, supporting complex classes, enumerations,
  structs, collections, and arrays, Marcus Deecke, The Code Project. I use the
  three units to manage settings. Updated: 15 Apr 2007

  \item Mastering Delphi 2005, Marco Cant$\grave{u}$, Sybex. I follow Marco's
  tip to do XSL tranform in .NET.

  \item How to prevent multiple application instances?, Sergey A. Kryukov, BDN
  Code Central. I use this to ensure that only one instance of Utilities Plus
  executable is available.

  % \item A C\# auto complete combo box, Matt Berther, The Code Project. I add one
  % line and use this in Source Navigator.
  \item Line Counter - Writing a Visual Studio 2005 Add-In, Jon Rista, The Code
  Project. 	Updated: 31 May 2007

  \item Line Counter - Writing a SharpDevelop AddIn, Daniel Grunwald, The Code
  Project. Updated: 18 Jul 2006
\end{itemize}

\appendix
\section{Things on SBT source code: A letter from SBT's author David}
\label{sec:sbt}
This section contains the reply from David Hervieux, the original author of
SBT. All files used in CBC in the namespace "BeWise" are taken from his wizard
either directly or with a few lines modified. In this reply, he permits me to
use his code and distribute the code with CBC.


\begin{quote}

  %\begin{flushleft}
Date: Sun, 09 Oct 2005 16:19:20 -0400 \\
Sender: "David Hervieux" $<$dhervieux@videotron.ca$>$\\
Title: Re: [dev] Ask you about SBT license issue \\
Receiver: cylextm-guard@yahoo.com.cn
%\end{flushleft}
\end{quote}
\begin{quotation}
Hi,

  I'm no longer working on Sharp Builder Tools. Unfortunatly I got too many
  problem with Borland. But feel free to uses the code in Sharp Builder Tools.
  It will be a pleasure to me if somebody uses it. I did the product to learn
  C\# so I don't really care about the license, so don't be shy to incoporate
  in your expert all the features that you need from SBT. You can't also change
  the code (and the namespace) to reflect your add in.

  David

\end{quotation}

\section{Things on bundled tools}

Some tools are bundled in bundled folder or other places.

\subsection{Artistic Style (written by Tal Davidson)} Artistic Style is a
reindenter and reformatter of C, C++, C\# and Java source code.

Artistic Style may be used and distributed under the GNU General Public
License (GPL).

Available on the web at

\hyperlink{http://astyle.sourceforge.net}

\subsection{JEDI Code Format (written by Anthony Steele)} The formatter can
standardise all aspects of Delphi Object Pascal source code formatting,
including indentation, spacing and capitalisation.

This program is free and open-source. It is covered by the Mozilla Public
Licence (MPL). I chose this particular open-source license at the suggestion of
the Delphi-JEDI group.

% \subsection{FavouritesMenu (written by Marc Rohloff)}
% I often find that the recent files list in C\# Builder no longer contains
% the files which I most often use. This add-in as an update of the previous
% version which I had created for Delphi 5. It adds a 'Favourites' menu
% option in the 'File' menu. But you can make it add a 'Favorites' menu option
% if you prefer it that way.
%
% Needless to say this software is freeware and comes with no warranties against defects or any undesirable side-effects.

\subsection{DILMerge (written by Mauro Venturini)}
Delphi tradition was to generate a single executable but when strong reasons
requested separated run time packages. With .NET things are different: a
typical .NET application consists of an executable assembly, a bunch of
assemblies in the program directory, and a bunch of assemblies in the GAC. Now,
ok for the GAC, we have been used to system DLLs for a long time, but that DLLs
around the executable look so VBish...

Really MS has a assembly linker, ILMerge, that can be downloaded at (please
read ILMerge.dll.txt)
http://www.microsoft.com/down\-loads/de\-tails.aspx?Fa\-mi\-ly\-ID=22914587-b4ad-4eae-87cf-b14ae6a939b0\&displaylang=en

ILMerge is a command line tool and its use is a bit akward, so I made a Delphi
oriented GUI for it: DILMerge.

DILMerge gets a Delphi or C\# project, extracts and shows a list of assembly
references and suggests which merge together. If necessary the list can be
adjusted and the ILMerge options tweaked before merging the assemblies and
generating a good old single file executable. Alternatively the list can be
manually constructed. When a project is available the options and the merge
assembly list can be saved inside it and automatically recovered the next time
the project is open.

\subsection{AddMany (written by Mauro Venturini)}
\label{sec:addmany}
When a project has many files the IDE Project Manager dialogs may be quite
cumbersome. AddMany introduces multiselection dialogs that make adding and
opening units very fast and easy.

Available on the web at
\hyperlink{http://cc.borland.com/Item.aspx?id=23368}

\subsection{Clearer (written by Mauro Venturini)}
History is a great new feature of Delphi 2005. There is only a little drawback:
after project completion getting rid of all the history files is a bit
annoying. Clearer makes this more easy.

Copyright (C) 2004 by Mauro Venturini.

\subsection{Icon Browser (written by Kenny Kerr)}
Created by Kenny Kerr in 2001. The bundled version is slightly modified by Lex
Mark.

Available on the web at
\hyperlink{http://www.kennyandkarin.com/Kenny/CodeCorner/Tools/IconBrowser/}

\subsection{Process Checker (written by David Clegg)}
Utility to monitor whether specified processes are running, and restart them if
necessary.

Dependant processes can be associated with each monitored process, and these
dependants will be terminated vefore the monitored process is restarted.

\subsection{ThirdParty.dll}
\label{sec:thirdparty}
FormXmlDocPreview unit is in public domain. See its unit header for details.

% AutoCompleteComboBox unit is created by Matt Berther.

\subsection{Tip of the Day}
This tool is a C\# port of CnPack's TipOfDay unit. Since the original code is
covered by CnPack IDE Wizards License, I put this piece of software under that
license, too. See that license in CnPack.txt in doc folder.

 Using VCL classes in WinForms is not my idea. I read that from a BDN Code
 Central entry \#22691. Its author is Bruce McGee. Thanks Bruce.


\subsection{Install, Plus2Plus2, and Templates Tools}
They are tools used in the installation process, or other purposes written by
me, \lextm. The code is public domain, you may use and modify it freely.

\end{document}
% Postamble Section
